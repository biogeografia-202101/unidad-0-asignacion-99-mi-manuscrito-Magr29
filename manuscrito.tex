\documentclass[11pt,]{article}
\usepackage[left=1in,top=1in,right=1in,bottom=1in]{geometry}
\newcommand*{\authorfont}{\fontfamily{phv}\selectfont}
\usepackage[]{mathpazo}


  \usepackage[T1]{fontenc}
  \usepackage[utf8]{inputenc}



\usepackage{abstract}
\renewcommand{\abstractname}{}    % clear the title
\renewcommand{\absnamepos}{empty} % originally center

\renewenvironment{abstract}
 {{%
    \setlength{\leftmargin}{0mm}
    \setlength{\rightmargin}{\leftmargin}%
  }%
  \relax}
 {\endlist}

\makeatletter
\def\@maketitle{%
  \newpage
%  \null
%  \vskip 2em%
%  \begin{center}%
  \let \footnote \thanks
    {\fontsize{18}{20}\selectfont\raggedright  \setlength{\parindent}{0pt} \@title \par}%
}
%\fi
\makeatother




\setcounter{secnumdepth}{3}

\usepackage{longtable,booktabs}

\usepackage{graphicx,grffile}
\makeatletter
\def\maxwidth{\ifdim\Gin@nat@width>\linewidth\linewidth\else\Gin@nat@width\fi}
\def\maxheight{\ifdim\Gin@nat@height>\textheight\textheight\else\Gin@nat@height\fi}
\makeatother
% Scale images if necessary, so that they will not overflow the page
% margins by default, and it is still possible to overwrite the defaults
% using explicit options in \includegraphics[width, height, ...]{}
\setkeys{Gin}{width=\maxwidth,height=\maxheight,keepaspectratio}

\title{Inventario de la familia Arecaceae en la paracela permanente de 50ha,
Barro Colorado Island.\\
Estudio de comunidad, biodiversidad, agrupamiento y asociación.\\
Ecología numérica con R.  }



\author{\Large Marcos Antonio González Reyes\vspace{0.05in} \newline\normalsize\emph{Estudiante, Universidad Autónoma de Santo Domingo (UASD)}  }


\date{}

\usepackage{titlesec}

\titleformat*{\section}{\normalsize\bfseries}
\titleformat*{\subsection}{\normalsize\itshape}
\titleformat*{\subsubsection}{\normalsize\itshape}
\titleformat*{\paragraph}{\normalsize\itshape}
\titleformat*{\subparagraph}{\normalsize\itshape}

\titlespacing{\section}
{0pt}{36pt}{0pt}
\titlespacing{\subsection}
{0pt}{36pt}{0pt}
\titlespacing{\subsubsection}
{0pt}{36pt}{0pt}





\newtheorem{hypothesis}{Hypothesis}
\usepackage{setspace}

\makeatletter
\@ifpackageloaded{hyperref}{}{%
\ifxetex
  \PassOptionsToPackage{hyphens}{url}\usepackage[setpagesize=false, % page size defined by xetex
              unicode=false, % unicode breaks when used with xetex
              xetex]{hyperref}
\else
  \PassOptionsToPackage{hyphens}{url}\usepackage[unicode=true]{hyperref}
\fi
}

\@ifpackageloaded{color}{
    \PassOptionsToPackage{usenames,dvipsnames}{color}
}{%
    \usepackage[usenames,dvipsnames]{color}
}
\makeatother
\hypersetup{breaklinks=true,
            bookmarks=true,
            pdfauthor={Marcos Antonio González Reyes (Estudiante, Universidad Autónoma de Santo Domingo (UASD))},
             pdfkeywords = {Arecaceae, BCI, Lago Gatún, Panamá},  
            pdftitle={Inventario de la familia Arecaceae en la paracela permanente de 50ha,
Barro Colorado Island.\\
Estudio de comunidad, biodiversidad, agrupamiento y asociación.\\
Ecología numérica con R.},
            colorlinks=true,
            citecolor=blue,
            urlcolor=blue,
            linkcolor=magenta,
            pdfborder={0 0 0}}
\urlstyle{same}  % don't use monospace font for urls

% set default figure placement to htbp
\makeatletter
\def\fps@figure{htbp}
\makeatother

\usepackage{pdflscape} \newcommand{\blandscape}{\begin{landscape}}
\newcommand{\elandscape}{\end{landscape}}


% add tightlist ----------
\providecommand{\tightlist}{%
\setlength{\itemsep}{0pt}\setlength{\parskip}{0pt}}

\begin{document}
	
% \pagenumbering{arabic}% resets `page` counter to 1 
%
% \maketitle

{% \usefont{T1}{pnc}{m}{n}
\setlength{\parindent}{0pt}
\thispagestyle{plain}
{\fontsize{18}{20}\selectfont\raggedright 
\maketitle  % title \par  

}

{
   \vskip 13.5pt\relax \normalsize\fontsize{11}{12} 
\textbf{\authorfont Marcos Antonio González Reyes} \hskip 15pt \emph{\small Estudiante, Universidad Autónoma de Santo Domingo (UASD)}   

}

}








\begin{abstract}

    \hbox{\vrule height .2pt width 39.14pc}

    \vskip 8.5pt % \small 

\noindent Se obtubieron datos de abundancia global por cuadrante de las especies
de las familias presentes en BCI, con los cuadrantes 3, 5, 20, 43 y 48
mostrando la mayor abundancia de individuos; y los cuadrantes 22, 23,
31, 34 y 45 siendo los de menor abundancia. La distribución de pH en la
parcela se presenta de izquierda a derecha, hacia la izquierda se
encuentran los valores de pH más bajos y hacia la derecha los más altos.
Para la asociación de especies, los datos arrojaron que
\emph{Chamaedorea tepejilote} y \emph{Bactris barronis} son las dos
especies que no se asocian con otras especies. Existen asociaciones
entre variables geomorfologicas y entre variables de riqueza-abundancia.
La abundancia de la familia Arecaceae se encuentra asociada con la
riqueza de especies de la familia, la orientación media y la riqueza
global de especies de la parcela permanente. La riqueza de la familia se
asocia a su vez con la variable geomorfológica de pico. Mediante
técnicas de agrupamiento se lograron alcanzar 5 grupos de sitios
mediante la técnica Ward. Utilizando el Coeficiente de correlación
biserial puntual se observa que el grupo dos de la técnica de
agrupamiento Complete mantiene a \emph{Socratea exorrizha} como especie
indicadora. El I de Moran con la transformada de Hellinger con y sin
tendencia, muestra que \emph{Bactris major} tiene una dependencia
espacial con ciertas variables, como Ca, B y Mg, donde bajos valores de
estas variables llevan a que haya abundancia significativa de esta
especie en sus sitios de muestreo.


\vskip 8.5pt \noindent \emph{Keywords}: Arecaceae, BCI, Lago Gatún, Panamá \par

    \hbox{\vrule height .2pt width 39.14pc}



\end{abstract}


\vskip 6.5pt


\noindent  \section{Introducción}\label{introducciuxf3n}

Barro Colorado es una isla localizada en el lago Gatún del Canal de
Panamá. Es un área protegida, en la cual se esstudian los bosques
tropicales y que junto con otras cinco penínsulas cercanas, forma el
Monumento Natural Barro Colorado. Estructurado en 1923 y está
administrado por el Instituto Smithsoniano desde 1946.

Forma parte de The Center for Tropical Science, una red compuesta por
alrededor de 15 países, que estudian los bosques tropicales y la
metología estandarizada en grandes parcelas permanentes, siendo Barro
Colorado la primer gran parcela en ser establecida, censada por primera
vez en los años 1981-1983 (Condit, 1998). Esto con el fin de recolectar
y analizar datos ecológicos para monitorear la dinámica de poblaciones y
la diversidad en localidades permanentes a largo plazo.

Esto se conoce como Long-Term Monitoring, puede ser definido como el
levantamiento de datos durante determinado período de tiempo en áreas
contaminadas o con un alto índice de pérdida de especies. En palabras de
Lindenmayer \& Likens (2010), son las mediciones empíricas repetidas
basadas en el campo, se recopilan continuamente y luego se analizan
durante al menos 10 años.

Es un estudio a largo plazo cuando documenta los procesos importantes
que componen el ecosistema o el tiempo de generación del organismo
dominante, así, su duración se mide con la velocidad dinámica del
sistema que se está estudiando (Franklin, 1989).

En este estudio se estará trabajando con la familia de plantas
Arecaceae, comúnmente conocidas como palmeras. Las palmas o palmeras son
plantas monocotiledóneas y presentan un crecimiento apical, que al
perder esta parte, detienen su crecimiento, secan y mueren. Las palmas
cuentan con aproxiamadamente 3,500 especies, cuya mayoría se ve
representada por árboles, también se encuentran arbustos y enredaderas.
Las formas arbóreas no presentan ramificación en el tronco y tienen una
corona terminal en hojas, generalmente llamadas frondas, que emergen de
una en una de la yema apical. La lignina de sus paredes celulares le
confiere al tallo gran rigidez, y como monocotiledóneas no poseen
crecimiento secundario. Algunas de ellas tienen valor económico y pueden
se fuente de alimento, refugio, ropa y combustible (Glimn-Lacy \&
Kaufman, 2006).

\section{Metodología}\label{metodologuxeda}

\subsection{Área de estudio}\label{uxe1rea-de-estudio}

El área de estudio es la parcela de 50ha en Isla Barro Colorado,
localizada en el Lago Gatún en Panamá (Hubbell, Condit, \& Foster, 2005)
(ver figura \ref{fig:mapa_cuadros}). Cuenta con una extensión de terreno
de 54km\^{}2, cada hectarea tiene 1km\^{}2; la familia de plantas a
examinar es Arecaceae Schultz Sch. Se fundamentó en el uso del censo de
BCI, tomando en cuenta una matriz de comunidad que representa las
especies censadas. Cuenta con al menos 315 especies identificadas
(Condit, 1998). Los datos levantados para calificar un individuo como
apto para el censo se basaron en el diámetro a la altura del pecho (DAP)
de 1cm, los censos por lo general se toman un período de 5 años para ser
actualizados, en cada censo se colectan datos para ver que tanto han
crecido los individuos, evaluar si hay nuevas especies y actualizar el
número de individuos que cumplen con los requerimientos para ser
anotados y enlistados en dicho censo.

\begin{figure}
\centering
\includegraphics[width=0.50000\textwidth]{mapa_cuadros.png}
\caption{Área de estudio, parcela de 50ha, Isla Barro Colorado.
\label{fig:mapa_cuadros}}
\end{figure}

\subsection{Materiales y métodos}\label{materiales-y-muxe9todos}

Para este estudio se utilizó el software estadístico R (R Core Team,
2020), los paquetes vegan (Oksanen et al., 2019), tidyverse (Wickham,
2017), sf(Pebesma, 2018), mapview(Appelhans, Detsch, Reudenbach, \&
Woellauer, 2019), leaflet (Cheng, Karambelkar, \& Xie, 2018),
ez(Lawrence, 2016), psych(Revelle, 2019), adespatial (Dray et al.,
2020), stats (R Core Team, 2020) y scripts del repositorio ``Scripts de
análisis de BCI'' (Batlle, 2020).

Se utiliza un análisis de medición de asociación buscando calcular el
grado de asociación entre especies, sitios y variables, para ello se
utilizan índices de similaridad, disimilaridad y distancia, usando los
de Jaccard, Sorensen, índice rho de Pearson, y Spearman, también la
transformada de Chi para comparar correlación entre especies. Se usan
los modos Q y R, Q para la disimilaridad entre obejetos (sitios) usando
la abundancia de individuos en una matriz de distancia Euclídea y la
transformada de Hellinger, que divide los valores de una fila por la
suma de la fila y toma como resultado la raíz cuadrada de los valores
obtenidos, la función `coldiss' para generar mapas de calor con los
sitios que forman grupos cercanos o lejanos.

También se usaron modelos con enfoques asintóticos: Modelo homogéneo
(estándar y MLE), Chao1 y Chao1-bc, iChao1, Basados en ``cobertura'' o
``completitud de muestra''. ACE para datos de abundancia, Estimadores
Jackknife (de primer y segundo órdenes), que estiman la riqueza de
especies; y modelos no asintóticos para rarefacción y extrapolación de
datos.

En modo R se calculó la correlación entre espescies, usando dotos
cuantitativos de especies (riqueza), y entre variables para obtener el
grado de asociación aplicando la transformación de Chi para comparar
especies e índice rho de Spearman y método de Pearson para las variables
ambientales.

Se realizaron análisis de agrupamiento utilizando técnicas de
agrupamiento jerárquico, para agrupar según características o
preferencias comúnes entre sitios, aquí se utilizan criterios de enlace
y sus resultados se visualizan en dendrograma, para cuya elaboración se
usó distancia Euclídea, que es la distancia entre un punto y otro según
variables descriptivas y se usa la distancia mínima. Los criteros de
enlace utilizados aquí fueron de enlace completo y ward de varianza
mínima. Para la generación de dendrogramas se utilizó la función
`hclust' del paquete `stats', presente por defecto en R. A su vez se
utilizó `anchura de silueta' para determinar cuantos grupos serian
necesarios o mas representativos, pruebas multivariadas de Bootstrap en
busqueda de los sitios que presentaran altas probabilidades de formar
grupos, y la funcion `heatmap' en la obtencion de mapas de calor para
representar grupos de sitio y la distancia entre especies.

Para calcular el grado de diversidad alpha usamos los grupos generados
mediante el análisis de agrupamiento, los `grupos\_complete\_k2' y
`grupos\_ward\_k5'. A fin de calcular esto se utilizan métodos como la
Entropia de Shannon \emph{H1}, que calcula el grado de desorden en la
muestra, índice de concentración de Simpson, que calcula la probabilidad
de que dos individuos seleccionados aleatoriamente puedan ser de la
misma especie, Equidad de Pielou \emph{J=H1/H0}. Números de Hill,
riqueza de especies \emph{N0=q}, número de especies abundantes
\emph{N1=exp(H)}, y el inverso de Simpson \emph{N1=D}; esto para
calcular diversidad alpha. Para diversidad beta calculamos la
contribución de las especies y los sitios a la diversidad beta, usando
la transformación de Hellinger.

En ecología espacial usamos la transformada de hellinger y la matriz
ambiental para crear un cuadro de vecindad y ver como se
autocorrelacionan los sitios, se genera un correlograma para la variable
que queremos estudiar mediante la función `sp.correlogram' y para varias
variables como la abundancia de especies y variables ambientales,
también usando otros métodos como la prueba Mantel con matrices de
distancia para autocorrelación espacial con y sin tendencia, y el I de
Moran con una matriz de abundancia de especies transformada sin
tendencia y se aplica a variables ambientales para obtener los datos de
autocorrelación y distribución de especies y variables en los sitios de
muestreo.

\section{Resultados}\label{resultados}

Se muestran los resultados del estudio realizado a la familia Arecaceae
en la parcela de 50ha de Barro Colorado.

\subsection{Riqueza-abundancia y
presecia-ausencia}\label{riqueza-abundancia-y-presecia-ausencia}

En la Tabla \ref{tab:abun_sp} se muestran las especies presentes en la
parcela permanete y la abundancia de individuos de cada especie, siendo
Oenocarpus mapora la especie con el mayor número de individuos y
Chamaedorea tepejilote la especie con el menor número de individuos; en
la figura \ref{fig:abun_sp_q} vemos la distribución de las especies en
la parcela, la cantidad de individuos de las especies presentes en cada
sitio y cuales especies tienen mayor o menor presencia en el área de
estudio, donde Oenocarpus mapora es la especie que mayor presencia
presenta, se encuentra en todos los sitios muestreados, y Elaeis
oleifera la especie con menor presencia, solo encontrada en tres sitios
de muestreo.

Se obtubieron datos de abundancia global por cuadrante de las especies
de las familias presentes en BCI, con los cuadrantes 3, 5, 20, 43 y 48
mostrando la mayor abundancia de individuos; y los cuadrantes 22, 23,
31, 34 y 45 siendo los de menor abundancia, figura
\ref{fig:mapa_cuadros_abun_global}. Los cuadrantes que mostraron una
mayor riqueza de especies a nivel global fueron los cuadrantes 1, 27 y
50 con 181, 195 y 193 especies respectivamente; los de menor riqueza
fueron los cuadrantes 34, 35 y 41 con 138, 147 y 146 especies
respectivamente, figura \ref{fig:mapa_cuadros_riq_global}.

La abundancia de individuos para la familia Arecaceae se presenta en la
figura \ref{fig:mapa_cuadros_abun_mi_familia}, donde observamos que los
sitios/cuadrantes que mayor abundancia presentaron fueron los sitios 5,
6 y 27, con 113, 171 y 136 individuos respectivamente; y los sitios 35,
40 y 41 los de menor abundancia, presentando 9, 15 y 10 individuos
respectivamente. Los cuadrantes que mayor riqueza mostraron tuvieron un
numero de cinco y seis especies, los que menos mostraron tuvieron dos y
tres especies, ningun cuadrante mostro una sola especie, asi mismo
ninguno presento la totalidad de especies presentes en la parcela, cuyo
valor es de nueve especies (ver figura
\ref{fig:mapa_cuadros_riq_mi_familia}).

La distribución de pH en la parcela se presenta de izquierda a derecha,
hacia la izquierda se encuentran los valores de pH más bajos y hacia la
derecha los más altos (ver figura \ref{fig:mapa_cuadros_pH}).

\begin{longtable}[]{@{}lr@{}}
\caption{\label{tab:abun_sp}Abundancia de individiuos por especie de la
familia Arecaceae.}\tabularnewline
\toprule
Latin & n\tabularnewline
\midrule
\endfirsthead
\toprule
Latin & n\tabularnewline
\midrule
\endhead
Oenocarpus mapora & 1802\tabularnewline
Socratea exorrhiza & 500\tabularnewline
Astrocaryum standleyanum & 152\tabularnewline
Bactris major & 112\tabularnewline
Attalea butyracea & 32\tabularnewline
Elaeis oleifera & 20\tabularnewline
Bactris coloniata & 10\tabularnewline
Bactris barronis & 5\tabularnewline
Chamaedorea tepejilote & 4\tabularnewline
\bottomrule
\end{longtable}

\begin{figure}
\centering
\includegraphics{manuscrito_files/figure-latex/unnamed-chunk-3-1.pdf}
\caption{\label{fig:abun_sp_q}Abundancia de individuos por especie en
cada cuadrante.}
\end{figure}

\begin{figure}
\centering
\includegraphics[width=0.50000\textwidth]{mapa_cuadros_abun_global.png}
\caption{Mapa de abundancia global.
\label{fig:mapa_cuadros_abun_global}}
\end{figure}

\begin{figure}
\centering
\includegraphics[width=0.50000\textwidth]{mapa_cuadros_riq_global.png}
\caption{Mapa de riqueza global. \label{fig:mapa_cuadros_riq_global}}
\end{figure}

\begin{figure}
\centering
\includegraphics[width=0.50000\textwidth]{mapa_cuadros_abun_mi_familia.png}
\caption{Mapa de abundancia de individuos por cuadrante de la familia
Arecaceae. \label{fig:mapa_cuadros_abun_mi_familia}}
\end{figure}

\begin{figure}
\centering
\includegraphics[width=0.50000\textwidth]{mapa_cuadros_riq_mi_familia.png}
\caption{Mapa de riqueza de especies de la familia Arecaceae.
\label{fig:mapa_cuadros_riq_mi_familia}}
\end{figure}

\begin{figure}
\centering
\includegraphics[width=0.50000\textwidth]{mapa_cuadros_ph.png}
\caption{Mapa de pH, parcela de 50ha. \label{fig:mapa_cuadros_pH}}
\end{figure}

\subsection{Medición de asociación}\label{mediciuxf3n-de-asociaciuxf3n}

\subsubsection{Asociacion de sitios}\label{asociacion-de-sitios}

Se utilizó la matriz de comunidad de la familia para generar una matriz
de distancia Euclídea mediante la transformación de Hellinger, para
obtener los datos de asociación entre los sitios muestreados. Estos
datos en la matriz de disimilaridad oredenada, a la derecha, reflejan
que existen al menos tres grandes grupos altamente asociados, mostrados
en color fucsia, y en color cian los de baja asociación (ver figura
\ref{fig:matriz_disimilaridad_hellinger}).

\begin{figure}
\centering
\includegraphics{matriz_disimilaridad_hellinger.png}
\caption{Matriz de correlacion entre sitios.
\label{fig:matriz_disimilaridad_hellinger}}
\end{figure}

\subsubsection{Asociacion de especies}\label{asociacion-de-especies}

Para la asociación de especies, los datos arrojaron que
\emph{Chamaedorea tepejilote} y \emph{Bactris barronis} son las dos
especies que no se asocian con otras especies; \emph{Bactris coloniata}
no presenta asociación con \emph{Chamaedorea tepejilote}, \emph{Elaeis
oleifera} ni \emph{Bactris major}; \emph{Socratea exorrizha} no se
asocia con \emph{Chamaedorea tepejilote}, \emph{Bactris barronis} ni
\emph{Elaeis oleifera}, pero sí se asocia con \emph{Bactris coloniata},
\emph{Oenocarpus mapora}, \emph{Astrocaryum standleyanum}, \emph{Attalea
butyracea} y \emph{Bactris major}; \emph{Oenocarpus mapora} no tiene
asociación con las especies \emph{Chamaedorea tepejilote} ni
\emph{Bactris barronis}, mientras que sí presenta asociaciones con las
demás especies; el mismo caso anterior se repite con las especies
\emph{Astrocaryum standleyanum} y \emph{Attalea butyracea}; Elaeis
oleifera no presenta asociaciones con \emph{Chamaedorea tepejilote},
\emph{Bactris barronis}, \emph{Bactris coloniata} ni \emph{Socratea
exorrizha}, pero sí con las otras especies; \emph{Bactris major} carece
de asociación con las especies \emph{Chamaedorea tepejilote},
\emph{Bactris barronis} y \emph{Bactris coloniata}. Esto se puede
visualizar en la figura \ref{fig:matriz_asociacion_especies}, donde los
colores cian significan nula asociación, y los colores fucsia la
asociación que presentan las especies.

En la figura \ref{fig:matriz_distancia_especies} se pueden leer los
resultados de la distancia entre las especies, donde los colores cian
denotan larga distancia Euclídea, y los colores fucsia la corta
distancia.

\begin{figure}
\centering
\includegraphics{matriz_asociacion_especies.png}
\caption{Matriz de asociacion entre especies.
\label{fig:matriz_asociacion_especies}}
\end{figure}

\begin{figure}
\centering
\includegraphics{matriz_distancia_especies.png}
\caption{Matriz de distancia entre especies.
\label{fig:matriz_distancia_especies}}
\end{figure}

\subsubsection{Asociacion de variables}\label{asociacion-de-variables}

Existen asociaciones entre variables geomorfologicas y entre variables
de riqueza-abundancia. La abundancia de la familia Arecaceae se
encuentra asociada con la riqueza de especies de la familia, la
orientación media y la riqueza global de especies de la parcela
permanente. La riqueza de la familia se asocia a su vez con la variable
geomorfológica de pico. Las variables geomorfológicas que mayor grado de
asociación presentaron fueron las de pendiente media con con llanura,
cuando un disminuía la otra aumentaba, creando de esta manera una
correlación inversa; elevación media con vaguada, espolón/gajo e
interfluvio mantiene una asociación inversa; hay asociación entre valle
y vaguada; vaguada y se asocia con espolón/gajo de forma positiva, y de
manera negativa con llanura; y hombrera se relaciona con llanura, (ver
figura \ref{fig:matriz_correlacion_geomorf_abun_riq_spearman}).

La abundancia de individuos estuvo asociada con nueve variables, de las
cuales destacan las variables pH, Nitrogeno, Zinc y Boro, asociadas de
de forma negativa. La variable pH estuvo asociada destacablemente con
Nitrogeno, Zinc, Magnesio, Potasio, y Boro de forma positiva, y de
manera negativa con el Aluminio. El grado mas alto de asociacion estuvo
entre el Calcio y el Magnesio, figura
\ref{fig:matriz_correlacion_suelo_abun_riq_spearman}.

\begin{figure}
\centering
\includegraphics{matriz_correlacion_geomorf_abun_riq_spearman.png}
\caption{Matriz de correlacion de variables geomorfologicas y
abundancia-riqueza.
\label{fig:matriz_correlacion_geomorf_abun_riq_spearman}}
\end{figure}

\begin{figure}
\centering
\includegraphics{matriz_correlacion_suelo_abun_riq_spearman.png}
\caption{Matriz de correlacion de variables suelo y abundancia-riqueza.
\label{fig:matriz_correlacion_suelo_abun_riq_spearman}}
\end{figure}

\subsection{Análisis de agrupamiento}\label{anuxe1lisis-de-agrupamiento}

En la figura \ref{fig:Cluster_Euclideandistance_Ward} se muestran los
sitios que tienen una alta probabilidad de formar grupos, representados
en rectangulos de color azul, donde se forma un gran grupo con 27
sitios, tres grupos de tamaño medio con 7, 6 y 5 sitios, y dos grupos
pequeños con 2 y 3 sitios cada uno. En el la figura
\ref{fig:mapa_ward_k5} se muestra la distribución de estos grupos dentro
de la parcela de 50ha de BCI. La distribución de los grupos es
aleatoria, los grupos no se presentan de manera continua dentro de la
parcela de 50ha de BCI, sino que, se visualizan de manera dispersa, esto
puede deberse a distintas variables.

En la figura \ref{fig:Grupos_Ward_k5_variables} se observa la asociación
de variables a la distribución de estos grupos, destacando las
asociaciones de las variables pH, Zinc, Magnesio, Manganeso, Fósforo,
Aluminio, Nitrógeno, Curvatura, Pendiente, UTM y Hombrera. Los sitios
pertenecientes al grupo cuatro tienen marcada asociaión con variables de
suelo como (Potasio, Hierro, Aluminio) mientras que con otras variables
de suelo tiene valores de asociación bajos; con variables
ambientales/geomorfológicas como (Riqueza global, Vertiente,
Heterogeneidad ambiental y Pendiente media) presenta altos grados de
dependencia.

Mediante el análisis de especies indicadoras de IndVal, se obtuvo que el
método de agrupamiento Complete con dos grupos, presentó una sola
especie indicadora, encontrada en el grupo dos, la cual posee una
abundancia de 500 individuos. Con el método Ward el grupo cuatro pesentó
como especies indicadoras a \emph{Elaeis oleifera}, \emph{Bactris major}
y \emph{Attalea butyracea}; con la suma de grupos 1+2 se encuentra
\emph{Socratea exorrizha}, y para los grupos 4+5 \emph{Astrocaryum
standleyanum}.

Utilizando el Coeficiente de correlación biserial puntual se observa que
el grupo dos de la técnica de agrupamiento Complete mantiene a
\emph{Socratea exorrizha} como especie indicadora. Mientras que para la
técnica Ward hay una variación, no hay especies que se vean asociadas a
dos grupos o más, solo a un grupo, siendo, \emph{Socratea exorrizha}
para el sitio dos; y \emph{Elaeis oleifera}, \emph{Bactris major} y
\emph{Attalea butyracea} para el grupo cuatro.

\begin{figure}
\centering
\includegraphics{Cluster_Euclideandistance_Ward.png}
\caption{Dendrograma de distancia Euclidea con remuestreo de Bootstrap.
\label{fig:Cluster_Euclideandistance_Ward}}
\end{figure}

\begin{figure}
\centering
\includegraphics[width=0.50000\textwidth]{mapa_ward_k5.png}
\caption{Mapa distribucion de sitios agrupados.
\label{fig:mapa_ward_k5}}
\end{figure}

\begin{figure}
\centering
\includegraphics{Grupos_Ward_k5_variables.png}
\caption{Grafico de correlacion entre valiables y grupos de sitios.
\label{fig:Grupos_Ward_k5_variables}}
\end{figure}

\subsection{Análisis de diversidad}\label{anuxe1lisis-de-diversidad}

Para la matriz combinada de especies de la familia los análisis de
diversidad de Chao y Jacknife muestran que la riqueza de especies está
completamente representada, ya que, el número de especies de la familia
presentes en BCI es de 9, mientras que el número de especies muestreadas
fue exactamente de 9 especies.

Para los grupos Ward estos estimadores, sugieren que en los grupos 1 y 3
es posible encontrar una especie adicional si se aumentara el esfuerzo
de muestreo; y los grupos 2, 4 y 5 muestran total representación de
especies de la familia en base a lo estimado por los análisis.

En diversidad alpha la Equidad de Pielou y la Ratios de Hill se ven
asociadas con la elevación media y abundancia global. Las coordenadas
UTM.NS se relacionan con la riqueza de especies (N0) de manera negativa,
y de manera positiva con las Ratios de Hill y la Equidad de Pielou. En
cuanto a diversidad beta, se encontró que tanto los sitios, como las
especies, contribuyen a esta diversidad, siendo los sitios 13 y 23 los
contribuyentes, y las especies \emph{Astrocaryum butyracea},
\emph{Socratea exorrizha} y \emph{Bactris major}.

\subsection{Análisis de Ecología
espacial}\label{anuxe1lisis-de-ecologuxeda-espacial}

El I de Moran señala que el pH muestra una autocorrelación espacial de
los ordenes 1 al 7. En tanto que para las especies, \emph{Astrocaryum
standleyanum} la tiene para los órdenes 1 y 2, y para los órdenes 5 y 6,
\emph{Elaeis oleifera} para orden 1, \emph{Oenocarpus mapora} de orden 1
, \emph{Socratea exorrizha} para orden 1 y 5. Hay autocorrelación para
algunas variables geomorfológicas, pero solo para 1 orden, como es el
caso de llanura, espolón/gajo, vertiente y vaguada. Para las variables
de suelo las más destacables son B, Ca, Zn y K en los órdenes (1-3 y
5-8); Mg órdenes (1-3 y 6-8); N (1-2 y 4-6); N.min (1-3 y 7-9); y pH
órdenes (1-7).

Para la prueba Mantel existe autocorrelación para el orden 3, para un
valor de 0.001.

El I de Moran con la transformada de Hellinger con y sin tendencia,
muestra que \emph{Bactris major} tiene una dependencia espacial con
ciertas variables, como Ca, B y Mg, donde bajos valores de estas
variables llevan a que haya abundancia significativa de esta especie en
sus sitios de muestreo. \emph{Elaeis oleifera} a su vez parece estar
asociada o en dependencia de las variables Ca, B, Mn, y en menor medida
con K, Cu, Zn, N, pH, cuando estos valores son bajos; y con Al cuando
son altos.

Debido al azar con el que se presentan las especies de la familia en el
área de estudio estimo que estos modelos no predicen con gran certeza la
distribución u ocurrencia de las mismas.

\section{Discusión}\label{discusiuxf3n}

Estudios de poblaciones de especies de la familia Arecaceae también
reportan que Socratea exorrizha es una especie que parece siempre
presentar una abundancia significativa de individuos como reporta
(Stevenson \& Rodríguez, 2008). \emph{Elaeis oleifera} es una indicadora
significativa de la del micro hábitat pantanoso (Legendre \& Condit,
2019) y reporta un aumento de abundancia de 1 individuo, actualmente ese
número ha incrementado a 20.

\emph{Bactris barronis} y \emph{Chamaedorea tepejilote} fueron las
especies que no se relacionaron con más especies, esto puede estar
determinado por baja abundancia de individuos que presentaron, ya que
son las 2 especies con la más baja abundancia, también podría estar
relacionado con su distribución espacial y algunas variables ambientales
como valores bajos de elevación media, y bajos valores de Al, Fe, Mn, y
altos valores de pH fomentan la abundancia de \emph{Chamaedorea
tepejilote}. La asociación de variables también demostró estar
correlacionada con la distribución de las especies dentro de la parcela.
Cuatro pares de especies presentan asociación significativa,
\emph{Bactris major} y \emph{Elaeis oleifera}, \emph{Attalea butyracea}
y \emph{E. oleifera}, \emph{Oenocarpus mapora} y \emph{Socratea
exorrizha}, y \emph{Astrocaryum standleyanum} con \emph{B. major}.

Los sitios pertenecientes al grupo cuatro tienen marcada asociaión con
variables de suelo como (Potasio, Hierro, Aluminio) mientras que con
otras variables de suelo tiene valores de asociación bajos; con
variables ambientales/geomorfológicas como (Riqueza global, Vertiente,
Heterogeneidad ambiental y Pendiente media) presenta altos grados de
dependencia. Esto nos da una mejor perspectiva de por qué este grupo
presenta esta distribución y especies muy particulares \emph{Attalea
butyracea}, \emph{Bactris major} y \emph{Elaeis oleifera}, especies que
no se presentan en sitios tan contiguos como son los sitios 13, 18 y 23.

Para los grupos Ward, sugieren que en los grupos 1 y 3 es posible
encontrar una especie adicional si se aumentara el esfuerzo de muestreo;
y los grupos 2, 4 y 5 muestran total representación de especies de la
familia en base a lo estimado por los análisis. Cabe destacar que los
estimadores de Shanon y Simpson tuvieron sus variaciones en muchos
casos, pero de todas formas, lo observado y lo estimado estuvo dentro de
los parámetros de confianza.

El I de Moran señala que el pH muestra una autocorrelación espacial de
los ordenes 1 al 7. En tanto que para las especies, \emph{Astrocaryum
standleyanum} la tiene positiva para los órdenes 1 y 2, y negativa para
los órdenes 5 y 6, \emph{Elaeis oleifera} positiva para orden 1,
\emph{Oenocarpus mapora} de orden 1 negativa, \emph{Socratea exorrizha}
para orden 1 (positiva) y 5 (negativa). Hay autocorrelación para algunas
variables geomorfológicas, pero solo para 1 orden, como es el caso de
llanura, espolón/gajo, vertiente y vaguada. Para las variables de suelo
las más destacables son B, Ca, Zn y K en los órdenes (1-3 y 5-8); Mg
órdenes (1-3 y 6-8); N (1-2 y 4-6); N.min (1-3 y 7-9); y pH órdenes
(1-7). Estos resultados describirían por qué estas especies son las más
abundantes, exceptuando por \emph{E. oleifera} que es una de la de menor
abundancia, pero que es una especie indicadora del ambiente pantanoso.

\section{Agradecimientos}\label{agradecimientos}

\section{Información de soporte}\label{informaciuxf3n-de-soporte}

\ldots

\section{\texorpdfstring{\emph{Script}
reproducible}{Script reproducible}}\label{script-reproducible}

\ldots

\section*{Referencias}\label{referencias}
\addcontentsline{toc}{section}{Referencias}

\hypertarget{refs}{}
\hypertarget{ref-MapView}{}
Appelhans, T., Detsch, F., Reudenbach, C., \& Woellauer, S. (2019).
\emph{Mapview: Interactive viewing of spatial data in r}. Retrieved from
\url{https://CRAN.R-project.org/package=mapview}

\hypertarget{ref-jose_ramon_martinez_batlle_2020_4402362}{}
Batlle, J. R. M. (2020). biogeografia-master/scripts-de-analisis-BCI:
Long coding sessions (Version v0.0.0.9000).
\url{https://doi.org/10.5281/zenodo.4402362}

\hypertarget{ref-Leaflet}{}
Cheng, J., Karambelkar, B., \& Xie, Y. (2018). \emph{Leaflet: Create
interactive web maps with the javascript 'leaflet' library}. Retrieved
from \url{https://CRAN.R-project.org/package=leaflet}

\hypertarget{ref-condit1998tropical}{}
Condit, R. (1998). \emph{Tropical forest census plots: Methods and
results from barro colorado island, panama and a comparison with other
plots}. Springer Science \& Business Media.

\hypertarget{ref-adespatial}{}
Dray, S., Bauman, D., Blanchet, G., Borcard, D., Clappe, S., Guenard,
G., \ldots{} Wagner, H. H. (2020). \emph{Adespatial: Multivariate
multiscale spatial analysis}. Retrieved from
\url{https://CRAN.R-project.org/package=adespatial}

\hypertarget{ref-franklin1989importance}{}
Franklin, J. F. (1989). Importance and justification of long-term
studies in ecology. In \emph{Long-term studies in ecology} (pp. 3--19).
Springer.

\hypertarget{ref-glimn2006palm}{}
Glimn-Lacy, J., \& Kaufman, P. B. (2006). Palm family (arecaceae).
\emph{Botany Illustrated: Introduction to Plants, Major Groups,
Flowering Plant Families}, 125--125.

\hypertarget{ref-Hubbell2005Barro}{}
Hubbell, S., Condit, R., \& Foster, R. (2005). \emph{Forest census plot
on barro colorado island}.
\url{http://ctfs.si.edu/webatlas/datasets/bci/}.

\hypertarget{ref-EZ}{}
Lawrence, M. A. (2016). \emph{Ez: Easy analysis and visualization of
factorial experiments}. Retrieved from
\url{https://CRAN.R-project.org/package=ez}

\hypertarget{ref-legendre2019spatial}{}
Legendre, P., \& Condit, R. (2019). Spatial and temporal analysis of
beta diversity in the barro colorado island forest dynamics plot,
panama. \emph{Forest Ecosystems}, \emph{6}(1), 1--11.

\hypertarget{ref-lindenmayer2010science}{}
Lindenmayer, D. B., \& Likens, G. E. (2010). The science and application
of ecological monitoring. \emph{Biological Conservation}, \emph{143}(6),
1317--1328.

\hypertarget{ref-VeganPack}{}
Oksanen, J., Blanchet, F. G., Friendly, M., Kindt, R., Legendre, P.,
McGlinn, D., \ldots{} Wagner, H. (2019). \emph{Vegan: Community ecology
package}. Retrieved from \url{https://CRAN.R-project.org/package=vegan}

\hypertarget{ref-sfpackage}{}
Pebesma, E. (2018). Simple Features for R: Standardized Support for
Spatial Vector Data. \emph{The R Journal}, \emph{10}(1), 439--446.
\url{https://doi.org/10.32614/RJ-2018-009}

\hypertarget{ref-RSoft}{}
R Core Team. (2020). \emph{R: A language and environment for statistical
computing}. Retrieved from \url{https://www.R-project.org/}

\hypertarget{ref-psych}{}
Revelle, W. (2019). \emph{Psych: Procedures for psychological,
psychometric, and personality research}. Retrieved from
\url{https://CRAN.R-project.org/package=psych}

\hypertarget{ref-stevenson2008determinantes}{}
Stevenson, P. R., \& Rodríguez, M. E. (2008). Determinantes de la
composición florística y efecto de borde en un fragmento de bosque en el
guaviare, amazonía colombiana. \emph{Colombia Forestal}, \emph{11},
5--17.

\hypertarget{ref-Tidyverse}{}
Wickham, H. (2017). \emph{Tidyverse: Easily install and load the
'tidyverse'}. Retrieved from
\url{https://CRAN.R-project.org/package=tidyverse}




\newpage
\singlespacing 
\end{document}
