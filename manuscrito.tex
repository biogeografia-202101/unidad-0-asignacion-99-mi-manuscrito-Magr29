\documentclass[11pt,]{article}
\usepackage[left=1in,top=1in,right=1in,bottom=1in]{geometry}
\newcommand*{\authorfont}{\fontfamily{phv}\selectfont}
\usepackage[]{mathpazo}


  \usepackage[T1]{fontenc}
  \usepackage[utf8]{inputenc}



\usepackage{abstract}
\renewcommand{\abstractname}{}    % clear the title
\renewcommand{\absnamepos}{empty} % originally center

\renewenvironment{abstract}
 {{%
    \setlength{\leftmargin}{0mm}
    \setlength{\rightmargin}{\leftmargin}%
  }%
  \relax}
 {\endlist}

\makeatletter
\def\@maketitle{%
  \newpage
%  \null
%  \vskip 2em%
%  \begin{center}%
  \let \footnote \thanks
    {\fontsize{18}{20}\selectfont\raggedright  \setlength{\parindent}{0pt} \@title \par}%
}
%\fi
\makeatother




\setcounter{secnumdepth}{3}



\title{Título\\
Subtítulo\\
Subtítulo  }



\author{\Large Marcos Antonio González Reyes\vspace{0.05in} \newline\normalsize\emph{Estudiante, Universidad Autónoma de Santo Domingo (UASD)}  }


\date{}

\usepackage{titlesec}

\titleformat*{\section}{\normalsize\bfseries}
\titleformat*{\subsection}{\normalsize\itshape}
\titleformat*{\subsubsection}{\normalsize\itshape}
\titleformat*{\paragraph}{\normalsize\itshape}
\titleformat*{\subparagraph}{\normalsize\itshape}

\titlespacing{\section}
{0pt}{36pt}{0pt}
\titlespacing{\subsection}
{0pt}{36pt}{0pt}
\titlespacing{\subsubsection}
{0pt}{36pt}{0pt}





\newtheorem{hypothesis}{Hypothesis}
\usepackage{setspace}

\makeatletter
\@ifpackageloaded{hyperref}{}{%
\ifxetex
  \PassOptionsToPackage{hyphens}{url}\usepackage[setpagesize=false, % page size defined by xetex
              unicode=false, % unicode breaks when used with xetex
              xetex]{hyperref}
\else
  \PassOptionsToPackage{hyphens}{url}\usepackage[unicode=true]{hyperref}
\fi
}

\@ifpackageloaded{color}{
    \PassOptionsToPackage{usenames,dvipsnames}{color}
}{%
    \usepackage[usenames,dvipsnames]{color}
}
\makeatother
\hypersetup{breaklinks=true,
            bookmarks=true,
            pdfauthor={Marcos Antonio González Reyes (Estudiante, Universidad Autónoma de Santo Domingo (UASD))},
             pdfkeywords = {Arecaceae, BCI},  
            pdftitle={Título\\
Subtítulo\\
Subtítulo},
            colorlinks=true,
            citecolor=blue,
            urlcolor=blue,
            linkcolor=magenta,
            pdfborder={0 0 0}}
\urlstyle{same}  % don't use monospace font for urls

% set default figure placement to htbp
\makeatletter
\def\fps@figure{htbp}
\makeatother

\usepackage{pdflscape} \newcommand{\blandscape}{\begin{landscape}}
\newcommand{\elandscape}{\end{landscape}}


% add tightlist ----------
\providecommand{\tightlist}{%
\setlength{\itemsep}{0pt}\setlength{\parskip}{0pt}}

\begin{document}
	
% \pagenumbering{arabic}% resets `page` counter to 1 
%
% \maketitle

{% \usefont{T1}{pnc}{m}{n}
\setlength{\parindent}{0pt}
\thispagestyle{plain}
{\fontsize{18}{20}\selectfont\raggedright 
\maketitle  % title \par  

}

{
   \vskip 13.5pt\relax \normalsize\fontsize{11}{12} 
\textbf{\authorfont Marcos Antonio González Reyes} \hskip 15pt \emph{\small Estudiante, Universidad Autónoma de Santo Domingo (UASD)}   

}

}








\begin{abstract}

    \hbox{\vrule height .2pt width 39.14pc}

    \vskip 8.5pt % \small 

\noindent Resumen del manuscrito


\vskip 8.5pt \noindent \emph{Keywords}: Arecaceae, BCI \par

    \hbox{\vrule height .2pt width 39.14pc}



\end{abstract}


\vskip 6.5pt


\noindent  \section{Introducción}\label{introducciuxf3n}

Barro Colorado es una isla localizada en el lago Gatún del Canal de
Panamá. Es un área protegida, en la cual se esstudian los bosques
tropicales y que junto con otras cinco penínsulas cercanas, forma el
Monumento Natural Barro Colorado. Estructurado en 1923 y está
administrado por el Instituto Smithsoniano desde 1946.

Forma parte de The Center for Tropical Science, una red compuesta por
alrededor de 15 países, que estudian los bosques tropicales y la
metología estandarizada en grandes parcelas permanentes, siendo Barro
Colorado la primer gran parcela en ser establecida, censada por primera
vez en los años 1981-1983 (Condit, 1998). Esto con el fin de recolectar
y analizar datos ecológicos a fin de monitorear la dinámica de
poblaciones y la diversidad en localidades permanentes a largo plazo.

Esto se conoce como Long-Term Monitoring, puede ser definido como el
levantamiento de datos durante determinado período de tiempo en áreas
contaminadas o con un alto índice de pérdida de especies. En palabras de
Lindenmayer \& Likens (2010), son las mediciones empíricas repetidas
basadas en el campo se recopilan continuamente y luego se analizan
durante al menos 10 años.

Es un estudio a largo plazo cuando documenta los procesos importantes
que componen el ecosistema o el tiempo de generación del organismo
dominante, así, su duración se mide con la velocidad dinámica del
sistema que se está estudiando (Franklin, 1989).

\section{Metodología}\label{metodologuxeda}

\ldots

\section{Resultados}\label{resultados}

\ldots

\section{Discusión}\label{discusiuxf3n}

\section{Agradecimientos}\label{agradecimientos}

\section{Información de soporte}\label{informaciuxf3n-de-soporte}

\ldots

\section{\texorpdfstring{\emph{Script}
reproducible}{Script reproducible}}\label{script-reproducible}

\ldots

\section*{Referencias}\label{referencias}
\addcontentsline{toc}{section}{Referencias}

\hypertarget{refs}{}
\hypertarget{ref-condit1998tropical}{}
Condit, R. (1998). \emph{Tropical forest census plots: Methods and
results from barro colorado island, panama and a comparison with other
plots}. Springer Science \& Business Media.

\hypertarget{ref-franklin1989importance}{}
Franklin, J. F. (1989). Importance and justification of long-term
studies in ecology. In \emph{Long-term studies in ecology} (pp. 3--19).
Springer.

\hypertarget{ref-lindenmayer2010science}{}
Lindenmayer, D. B., \& Likens, G. E. (2010). The science and application
of ecological monitoring. \emph{Biological Conservation}, \emph{143}(6),
1317--1328.




\newpage
\singlespacing 
\end{document}
