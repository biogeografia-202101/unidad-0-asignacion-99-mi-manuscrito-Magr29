\documentclass[11pt,]{article}
\usepackage[left=1in,top=1in,right=1in,bottom=1in]{geometry}
\newcommand*{\authorfont}{\fontfamily{phv}\selectfont}
\usepackage[]{mathpazo}


  \usepackage[T1]{fontenc}
  \usepackage[utf8]{inputenc}



\usepackage{abstract}
\renewcommand{\abstractname}{}    % clear the title
\renewcommand{\absnamepos}{empty} % originally center

\renewenvironment{abstract}
 {{%
    \setlength{\leftmargin}{0mm}
    \setlength{\rightmargin}{\leftmargin}%
  }%
  \relax}
 {\endlist}

\makeatletter
\def\@maketitle{%
  \newpage
%  \null
%  \vskip 2em%
%  \begin{center}%
  \let \footnote \thanks
    {\fontsize{18}{20}\selectfont\raggedright  \setlength{\parindent}{0pt} \@title \par}%
}
%\fi
\makeatother




\setcounter{secnumdepth}{3}

\usepackage{longtable,booktabs}

\usepackage{graphicx,grffile}
\makeatletter
\def\maxwidth{\ifdim\Gin@nat@width>\linewidth\linewidth\else\Gin@nat@width\fi}
\def\maxheight{\ifdim\Gin@nat@height>\textheight\textheight\else\Gin@nat@height\fi}
\makeatother
% Scale images if necessary, so that they will not overflow the page
% margins by default, and it is still possible to overwrite the defaults
% using explicit options in \includegraphics[width, height, ...]{}
\setkeys{Gin}{width=\maxwidth,height=\maxheight,keepaspectratio}

\title{Inventario de las poblaciones de la familia Arecaceae presentes en la
paracela permanente de 50ha, Barro Colorado.\\
Estudio poblacional, biodiversidad, agrupamiento y asociación.\\
Ecología numérica con R  }



\author{\Large Marcos Antonio González Reyes\vspace{0.05in} \newline\normalsize\emph{Estudiante, Universidad Autónoma de Santo Domingo (UASD)}  }


\date{}

\usepackage{titlesec}

\titleformat*{\section}{\normalsize\bfseries}
\titleformat*{\subsection}{\normalsize\itshape}
\titleformat*{\subsubsection}{\normalsize\itshape}
\titleformat*{\paragraph}{\normalsize\itshape}
\titleformat*{\subparagraph}{\normalsize\itshape}

\titlespacing{\section}
{0pt}{36pt}{0pt}
\titlespacing{\subsection}
{0pt}{36pt}{0pt}
\titlespacing{\subsubsection}
{0pt}{36pt}{0pt}





\newtheorem{hypothesis}{Hypothesis}
\usepackage{setspace}

\makeatletter
\@ifpackageloaded{hyperref}{}{%
\ifxetex
  \PassOptionsToPackage{hyphens}{url}\usepackage[setpagesize=false, % page size defined by xetex
              unicode=false, % unicode breaks when used with xetex
              xetex]{hyperref}
\else
  \PassOptionsToPackage{hyphens}{url}\usepackage[unicode=true]{hyperref}
\fi
}

\@ifpackageloaded{color}{
    \PassOptionsToPackage{usenames,dvipsnames}{color}
}{%
    \usepackage[usenames,dvipsnames]{color}
}
\makeatother
\hypersetup{breaklinks=true,
            bookmarks=true,
            pdfauthor={Marcos Antonio González Reyes (Estudiante, Universidad Autónoma de Santo Domingo (UASD))},
             pdfkeywords = {Arecaceae, BCI, Lago Gatún, Panamá},  
            pdftitle={Inventario de las poblaciones de la familia Arecaceae presentes en la
paracela permanente de 50ha, Barro Colorado.\\
Estudio poblacional, biodiversidad, agrupamiento y asociación.\\
Ecología numérica con R},
            colorlinks=true,
            citecolor=blue,
            urlcolor=blue,
            linkcolor=magenta,
            pdfborder={0 0 0}}
\urlstyle{same}  % don't use monospace font for urls

% set default figure placement to htbp
\makeatletter
\def\fps@figure{htbp}
\makeatother

\usepackage{pdflscape} \newcommand{\blandscape}{\begin{landscape}}
\newcommand{\elandscape}{\end{landscape}}


% add tightlist ----------
\providecommand{\tightlist}{%
\setlength{\itemsep}{0pt}\setlength{\parskip}{0pt}}

\begin{document}
	
% \pagenumbering{arabic}% resets `page` counter to 1 
%
% \maketitle

{% \usefont{T1}{pnc}{m}{n}
\setlength{\parindent}{0pt}
\thispagestyle{plain}
{\fontsize{18}{20}\selectfont\raggedright 
\maketitle  % title \par  

}

{
   \vskip 13.5pt\relax \normalsize\fontsize{11}{12} 
\textbf{\authorfont Marcos Antonio González Reyes} \hskip 15pt \emph{\small Estudiante, Universidad Autónoma de Santo Domingo (UASD)}   

}

}








\begin{abstract}

    \hbox{\vrule height .2pt width 39.14pc}

    \vskip 8.5pt % \small 

\noindent Resumen del manuscrito


\vskip 8.5pt \noindent \emph{Keywords}: Arecaceae, BCI, Lago Gatún, Panamá \par

    \hbox{\vrule height .2pt width 39.14pc}



\end{abstract}


\vskip 6.5pt


\noindent  \section{Introducción}\label{introducciuxf3n}

Barro Colorado es una isla localizada en el lago Gatún del Canal de
Panamá. Es un área protegida, en la cual se esstudian los bosques
tropicales y que junto con otras cinco penínsulas cercanas, forma el
Monumento Natural Barro Colorado. Estructurado en 1923 y está
administrado por el Instituto Smithsoniano desde 1946.

Forma parte de The Center for Tropical Science, una red compuesta por
alrededor de 15 países, que estudian los bosques tropicales y la
metología estandarizada en grandes parcelas permanentes, siendo Barro
Colorado la primer gran parcela en ser establecida, censada por primera
vez en los años 1981-1983 (Condit, 1998). Esto con el fin de recolectar
y analizar datos ecológicos para monitorear la dinámica de poblaciones y
la diversidad en localidades permanentes a largo plazo.

Esto se conoce como Long-Term Monitoring, puede ser definido como el
levantamiento de datos durante determinado período de tiempo en áreas
contaminadas o con un alto índice de pérdida de especies. En palabras de
Lindenmayer \& Likens (2010), son las mediciones empíricas repetidas
basadas en el campo, se recopilan continuamente y luego se analizan
durante al menos 10 años.

Es un estudio a largo plazo cuando documenta los procesos importantes
que componen el ecosistema o el tiempo de generación del organismo
dominante, así, su duración se mide con la velocidad dinámica del
sistema que se está estudiando (Franklin, 1989).

\section{Metodología}\label{metodologuxeda}

El área de estudio es la parcela de 50ha en Isla Barro
Colorado,localizada en el Lago Gatún en Panamá (Hubbell, Condit, \&
Foster, 2005) (ver figura \ref{fig:mapa_cuadros}). Cuenta con una
extensión de terreno de 54km\^{}2, cada hectarea tiene 1km\^{}2; la
familia de plantas a examinar es Arecaceae Schultz Sch. Para este
estudio se utilizó el software estadístico R (R Core Team, 2020), los
paquetes vegan (Oksanen et al., 2019), tidyverse (Wickham, 2017),
sf(Pebesma, 2018), mapview(Appelhans, Detsch, Reudenbach, \& Woellauer,
2019), leaflet (Cheng, Karambelkar, \& Xie, 2018), ez(Lawrence, 2016),
psych(Revelle, 2019), adespatial (Dray et al., 2020), y scripts del
repositorio ``Scripts de análisis de BCI'' (Batlle, 2020).

\begin{figure}
\centering
\includegraphics[width=0.50000\textwidth]{mapa_cuadros.png}
\caption{Área de estudio, parcela de 50ha, Isla Barro Colorado.
\label{fig:mapa_cuadros}}
\end{figure}

También se realizaron análisis exploratorios de datos, determinación de
presencia-ausencia y abundancia/riqueza de especies de la familia, para
observar en cuales sitios había mayor o menor número de especies y
cuales sitios presentaron las abundancias más altas o más bajas;
análisis de agrupamiento, para determinar cuales sitios compartían
similitudes y características que los pudieran identificar como grupo.
Mediciones de asociación, con el fin de explorar las variables
(geomorfologicas, ambientales, suelo, etc.) asociadas a distintas
especies y sitios; diversidad biológica para evaluar como están
estructuradas las comuniades en los sitios muestrados (R. Kindt \& Coe,
2005); y ecología espacial que nos permite evaluar la autocorrelación
inherente al área de estudio.

Se utilizó una matriz ambiental con los datos generales del área
muestreada, una tabla censal de la familia estudiada; índices de
Jaccard, transformada de Hellinger, Bootstrap, y distancia Euclidea para
el estudio de biodiversidad, agrupamiento y determinar que sitios tienen
las mayores probablidades de formar clusters.

R Core Team (2020), R es un lenguaje y un entorno para la computación y
los gráficos estadísticos, proporciona una amplia variedad de técnicas
estadísticas (modelado lineal y no lineal, pruebas estadísticas
clásicas, análisis de series de tiempo, clasificación, agrupamiento,
etc.) y técnicas gráficas, y es altamente extensible, uno de los puntos
fuertes de R es la facilidad con la que se pueden producir gráficos con
calidad de publicación bien diseñados, incluidos símbolos matemáticos y
fórmulas cuando sea necesario .

El paquete vegan proporciona herramientas para la ecología descriptiva
de comunidades. Tiene las funciones más básicas de análisis de
diversidad, ordenación de comunidades y análisis de disimilitud. La
mayoría de sus herramientas multivariadas también se pueden utilizar
para otros tipos de datos (Oksanen et al., 2019).

Con el paquete `leaflet' se pueden crear y personalizar mapas
interactivos usando la librería de JavaScript y el paquete `htmlwidgets'
(Cheng et al., 2018).

Paquete `ez', las funciones de este paquete tienen como objetivo
proporcionar una especificación simple, intuitiva y coherente de
análisis y visualización de datos; las funciones de visualización
también incluyen la visualización del diseño para la auditoría de datos
previos al análisis y la visualización de la matriz de correlación;
finalmente, este paquete incluye funciones para el análisis no
paramétrico, incluidas las pruebas de permutación y remuestreo bootstrap
(Lawrence, 2016).

Paquete `psych', utilizado para el análisis multivariado y la
construcción de escalas utilizando análisis factorial, análisis de
componentes principales, análisis de conglomerados y análisis de
confiabilidad, aunque otras proporcionan estadísticas descriptivas
básicas (Revelle, 2019).

\section{Resultados}\label{resultados}

Se muestran los resultados del estudio realizado a la familia Arecaceae
en la parcela de 50ha de Barro Colorado.

\subsection{Riqueza-abundancia y
presecia-ausencia}\label{riqueza-abundancia-y-presecia-ausencia}

En la Tabla \ref{tab:abun_sp} se muestran las especies presentes en la
parcela permanete y la abundancia de individuos de cada especie, siendo
Oenocarpus mapora la especie con el mayor número de individuos y
Chamaedorea tepejilote la especie con el menor número de individuos; en
la figura \ref{fig:abun_sp_q} vemos la distribución de las especies en
la parcela, la cantidad de individuos de las especies presentes en cada
sitio y cuales especies tienen mayor o menor presencia en el área de
estudio, donde Oenocarpus mapora es la especie que mayor presencia
presenta, se encuentra en todos los sitios muestreados, y Elaeis
oleifera la especie con menor presencia, solo encontrada en tres sitios
de muestreo.

Se obtubieron datos de abundancia global por cuadrante de las especies
de las familias presentes en BCI, con los cuadrantes 3, 5, 20, 43 y 48
mostrando la mayor abundancia de individuos; y los cuadrantes 22, 23,
31, 34 y 45 siendo los de menor abundancia, figura
\ref{fig:mapa_cuadros_abun_global}. Los cuadrantes que mostraron una
mayor riqueza de especies a nivel global fueron los cuadrantes 1, 27 y
50 con 181, 195 y 193 especies respectivamente; los de menor riqueza
fueron los cuadrantes 34, 35 y 41 con 138, 147 y 146 especies
respectivamente, figura \ref{fig:mapa_cuadros_riq_global}.

La abundancia de individuos para la familia Arecaceae se presenta en la
figura \ref{fig:mapa_cuadros_abun_mi_familia}, donde observamos que los
sitios/cuadrantes que mayor abundancia presentaron fueron los sitios 5,
6 y 27, con 113, 171 y 136 individuos respectivamente; y los sitios 35,
40 y 41 los de menor abundancia, presentando 9, 15 y 10 individuos
respectivamente. Los cuadrantes que mayor riqueza mostraron tuvieron un
numero de cinco y seis especies, los que menos mostraron tuvieron dos y
tres especies, ningun cuadrante mostro una sola especie, asi mismo
ninguno presento la totalidad de especies presentes en la parcela, cuyo
valor es de nueve especies (ver figura
\ref{fig:mapa_cuadros_riq_mi_familia}).

La distribución de pH en la parcela se presenta de izquierda a derecha,
hacia la izquierda se encuentran los valores de pH más bajos y hacia la
derecha los más altos (ver figura \ref{fig:mapa_cuadros_pH}).

\subsection{Medición de asociación}\label{mediciuxf3n-de-asociaciuxf3n}

\subsubsection{Asociacion de sitios}\label{asociacion-de-sitios}

Se utilizó la matriz de comunidad de la familia para generar una matriz
de distancia Euclídea mediante la transformación de Hellinger, para
obtener los datos de asociación entre los sitios muestreados. Estos
datos en la matriz de disimilaridad oredenada, a la derecha, reflejan
que existen al menos tres grandes grupos altamente asociados, mostrados
en color fucsia, y en color cian los de baja asociación (ver figura
\ref{fig:matriz_disimilaridad_hellinger}).

\subsubsection{Asociacion de especies}\label{asociacion-de-especies}

Para la asociación de especies, los datos arrojaron que
\emph{Chamaedorea tepejilote} y \emph{Bactris barronis} son las dos
especies que no se asocian con otras especies; \emph{Bactris coloniata}
no presenta asociación con \emph{Chamaedorea tepejilote}, \emph{Elaeis
oleifera} ni \emph{Bactris major}; \emph{Socratea exorrizha} no se
asocia con \emph{Chamaedorea tepejilote}, \emph{Bactris barronis} ni
\emph{Elaeis oleifera}, pero sí se asocia con \emph{Bactris coloniata},
\emph{Oenocarpus mapora}, \emph{Astrocaryum standleyanum}, \emph{Attalea
butyracea} y \emph{Bactris major}; \emph{Oenocarpus mapora} no tiene
asociación con las especies \emph{Chamaedorea tepejilote} ni
\emph{Bactris barronis}, mientras que sí presenta asociaciones con las
demás especies; el mismo caso anterior se repite con las especies
\emph{Astrocaryum standleyanum} y \emph{Attalea butyracea}; Elaeis
oleifera no presenta asociaciones con \emph{Chamaedorea tepejilote},
\emph{Bactris barronis}, \emph{Bactris coloniata} ni \emph{Socratea
exorrizha}, pero sí con las otras especies; \emph{Bactris major} carece
de asociación con las especies \emph{Chamaedorea tepejilote},
\emph{Bactris barronis} y \emph{Bactris coloniata}. Esto se puede
visualizar en la figura \ref{fig:matriz_asociacion_especies}, donde los
colores cian significan nula asociación, y los colores fucsia la
ascociación que presentan las especies.

En la figura \ref{fig:matriz_distancia_especies} se pueden leer los
resultados de la distancia entre las especies, donde los colores cian
denotan larga distancia Euclídea, y los colores fucsia la corta
distancia.

\subsubsection{Asociacion de variables}\label{asociacion-de-variables}

Existen asociaciones entre variables geomorfologicas y entre variables
de riqueza-abundancia. La abundancia de la familia Arecaceae se
encuentra asociada con la riqueza de especies de la familia, la
orientación media y la riqueza global de especies de la parcela
permanente. La riqueza de la familia se asocia a su vez con la variable
geomorfológica de pico. Las variables geomorfológicas que mayor grado de
asociación presentaron fueron las de pendiente media con con llanura,
cuando un disminuía la otra aumentaba, creando de esta manera una
correlación inversa; elevación media con vaguada, espolón/gajo e
interfluvio mantiene una asociación inversa; hay asociación entre valle
y vaguada; vaguada y se asocia con espolón/gajo de forma positiva, y de
manera negativa con llanura; y hombrera se relaciona con llanura, (ver
figura \ref{fig:matriz_correlacion_geomorf_abun_riq_spearman}).

La abundancia de individuos estuvo asociada con nueve variables, de las
cuales destacan las variables pH, Nitrogeno, Zinc y Boro, asociadas de
de forma negativa. La variable pH estuvo asociada destacablemente con
Nitrogeno, Zinc, Magnesio, Potasio, y Boro de forma positiva, y de
manera negativa con el Aluminio. El grado mas alto de asociacion estuvo
entre el Calcio y el Magnesio, figura
\ref{fig:matriz_correlacion_suelo_abun_riq_spearman}.

\begin{longtable}[]{@{}lr@{}}
\caption{\label{tab:abun_sp}Abundancia de individiuos por especie de la
familia Arecaceae.}\tabularnewline
\toprule
Latin & n\tabularnewline
\midrule
\endfirsthead
\toprule
Latin & n\tabularnewline
\midrule
\endhead
Oenocarpus mapora & 1802\tabularnewline
Socratea exorrhiza & 500\tabularnewline
Astrocaryum standleyanum & 152\tabularnewline
Bactris major & 112\tabularnewline
Attalea butyracea & 32\tabularnewline
Elaeis oleifera & 20\tabularnewline
Bactris coloniata & 10\tabularnewline
Bactris barronis & 5\tabularnewline
Chamaedorea tepejilote & 4\tabularnewline
\bottomrule
\end{longtable}

\begin{figure}
\centering
\includegraphics{manuscrito_files/figure-latex/unnamed-chunk-3-1.pdf}
\caption{\label{fig:abun_sp_q}Abundancia de individuos por especie en
cada cuadrante.}
\end{figure}

\begin{figure}
\centering
\includegraphics[width=0.50000\textwidth]{mapa_cuadros_abun_global.png}
\caption{Mapa de abundancia global.
\label{fig:mapa_cuadros_abun_global}}
\end{figure}

\begin{figure}
\centering
\includegraphics[width=0.50000\textwidth]{mapa_cuadros_riq_global.png}
\caption{Mapa de riqueza global. \label{fig:mapa_cuadros_riq_global}}
\end{figure}

\begin{figure}
\centering
\includegraphics[width=0.50000\textwidth]{mapa_cuadros_abun_mi_familia.png}
\caption{Mapa de abundancia de individuos por cuadrante de la familia
Arecaceae. \label{fig:mapa_cuadros_abun_mi_familia}}
\end{figure}

\begin{figure}
\centering
\includegraphics[width=0.50000\textwidth]{mapa_cuadros_riq_mi_familia.png}
\caption{Mapa de riqueza de especies de la familia Arecaceae.
\label{fig:mapa_cuadros_riq_mi_familia}}
\end{figure}

\begin{figure}
\centering
\includegraphics[width=0.50000\textwidth]{mapa_cuadros_ph.png}
\caption{Mapa de pH, parcela de 50ha. \label{fig:mapa_cuadros_pH}}
\end{figure}

\begin{figure}
\centering
\includegraphics{matriz_disimilaridad_hellinger.png}
\caption{Matriz de correlacion entre sitios.
\label{fig:matriz_disimilaridad_hellinger}}
\end{figure}

\begin{figure}
\centering
\includegraphics{matriz_asociacion_especies.png}
\caption{Matriz de asociacion entre especies.
\label{fig:matriz_asociacion_especies}}
\end{figure}

\begin{figure}
\centering
\includegraphics{matriz_distancia_especies.png}
\caption{Matriz de distancia entre especies.
\label{fig:matriz_distancia_especies}}
\end{figure}

\begin{figure}
\centering
\includegraphics{matriz_correlacion_geomorf_abun_riq_spearman.png}
\caption{Matriz de correlacion de variables geomorfologicas y
abundancia-riqueza.
\label{fig:matriz_correlacion_geomorf_abun_riq_spearman}}
\end{figure}

\begin{figure}
\centering
\includegraphics{matriz_correlacion_suelo_abun_riq_spearman.png}
\caption{Matriz de correlacion de variables suelo y abundancia-riqueza.
\label{fig:matriz_correlacion_suelo_abun_riq_spearman}}
\end{figure}

\section{Discusión}\label{discusiuxf3n}

\section{Agradecimientos}\label{agradecimientos}

\section{Información de soporte}\label{informaciuxf3n-de-soporte}

\ldots

\section{\texorpdfstring{\emph{Script}
reproducible}{Script reproducible}}\label{script-reproducible}

\ldots

\section*{Referencias}\label{referencias}
\addcontentsline{toc}{section}{Referencias}

\hypertarget{refs}{}
\hypertarget{ref-MapView}{}
Appelhans, T., Detsch, F., Reudenbach, C., \& Woellauer, S. (2019).
\emph{Mapview: Interactive viewing of spatial data in r}. Retrieved from
\url{https://CRAN.R-project.org/package=mapview}

\hypertarget{ref-jose_ramon_martinez_batlle_2020_4402362}{}
Batlle, J. R. M. (2020). biogeografia-master/scripts-de-analisis-BCI:
Long coding sessions (Version v0.0.0.9000).
\url{https://doi.org/10.5281/zenodo.4402362}

\hypertarget{ref-Leaflet}{}
Cheng, J., Karambelkar, B., \& Xie, Y. (2018). \emph{Leaflet: Create
interactive web maps with the javascript 'leaflet' library}. Retrieved
from \url{https://CRAN.R-project.org/package=leaflet}

\hypertarget{ref-condit1998tropical}{}
Condit, R. (1998). \emph{Tropical forest census plots: Methods and
results from barro colorado island, panama and a comparison with other
plots}. Springer Science \& Business Media.

\hypertarget{ref-adespatial}{}
Dray, S., Bauman, D., Blanchet, G., Borcard, D., Clappe, S., Guenard,
G., \ldots{} Wagner, H. H. (2020). \emph{Adespatial: Multivariate
multiscale spatial analysis}. Retrieved from
\url{https://CRAN.R-project.org/package=adespatial}

\hypertarget{ref-franklin1989importance}{}
Franklin, J. F. (1989). Importance and justification of long-term
studies in ecology. In \emph{Long-term studies in ecology} (pp. 3--19).
Springer.

\hypertarget{ref-Hubbell2005Barro}{}
Hubbell, S., Condit, R., \& Foster, R. (2005). \emph{Forest census plot
on barro colorado island}.
\url{http://ctfs.si.edu/webatlas/datasets/bci/}.

\hypertarget{ref-Biodiv}{}
Kindt, R., \& Coe, R. (2005). \emph{Tree diversity analysis. a manual
and software for common statistical methods for ecological and
biodiversity studies}. Retrieved from
\url{http://www.worldagroforestry.org/output/tree-diversity-analysis}

\hypertarget{ref-EZ}{}
Lawrence, M. A. (2016). \emph{Ez: Easy analysis and visualization of
factorial experiments}. Retrieved from
\url{https://CRAN.R-project.org/package=ez}

\hypertarget{ref-lindenmayer2010science}{}
Lindenmayer, D. B., \& Likens, G. E. (2010). The science and application
of ecological monitoring. \emph{Biological Conservation}, \emph{143}(6),
1317--1328.

\hypertarget{ref-VeganPack}{}
Oksanen, J., Blanchet, F. G., Friendly, M., Kindt, R., Legendre, P.,
McGlinn, D., \ldots{} Wagner, H. (2019). \emph{Vegan: Community ecology
package}. Retrieved from \url{https://CRAN.R-project.org/package=vegan}

\hypertarget{ref-sfpackage}{}
Pebesma, E. (2018). Simple Features for R: Standardized Support for
Spatial Vector Data. \emph{The R Journal}, \emph{10}(1), 439--446.
\url{https://doi.org/10.32614/RJ-2018-009}

\hypertarget{ref-RSoft}{}
R Core Team. (2020). \emph{R: A language and environment for statistical
computing}. Retrieved from \url{https://www.R-project.org/}

\hypertarget{ref-psych}{}
Revelle, W. (2019). \emph{Psych: Procedures for psychological,
psychometric, and personality research}. Retrieved from
\url{https://CRAN.R-project.org/package=psych}

\hypertarget{ref-Tidyverse}{}
Wickham, H. (2017). \emph{Tidyverse: Easily install and load the
'tidyverse'}. Retrieved from
\url{https://CRAN.R-project.org/package=tidyverse}




\newpage
\singlespacing 
\end{document}
