\documentclass[11pt,]{article}
\usepackage[left=1in,top=1in,right=1in,bottom=1in]{geometry}
\newcommand*{\authorfont}{\fontfamily{phv}\selectfont}
\usepackage[]{mathpazo}


  \usepackage[T1]{fontenc}
  \usepackage[utf8]{inputenc}



\usepackage{abstract}
\renewcommand{\abstractname}{}    % clear the title
\renewcommand{\absnamepos}{empty} % originally center

\renewenvironment{abstract}
 {{%
    \setlength{\leftmargin}{0mm}
    \setlength{\rightmargin}{\leftmargin}%
  }%
  \relax}
 {\endlist}

\makeatletter
\def\@maketitle{%
  \newpage
%  \null
%  \vskip 2em%
%  \begin{center}%
  \let \footnote \thanks
    {\fontsize{18}{20}\selectfont\raggedright  \setlength{\parindent}{0pt} \@title \par}%
}
%\fi
\makeatother




\setcounter{secnumdepth}{3}

\usepackage{longtable,booktabs}

\usepackage{graphicx,grffile}
\makeatletter
\def\maxwidth{\ifdim\Gin@nat@width>\linewidth\linewidth\else\Gin@nat@width\fi}
\def\maxheight{\ifdim\Gin@nat@height>\textheight\textheight\else\Gin@nat@height\fi}
\makeatother
% Scale images if necessary, so that they will not overflow the page
% margins by default, and it is still possible to overwrite the defaults
% using explicit options in \includegraphics[width, height, ...]{}
\setkeys{Gin}{width=\maxwidth,height=\maxheight,keepaspectratio}

\title{La familia Arecaceae en la parcela permanente de 50ha, Barro Colorado
Island.\\
Ecología de comunidades, biodiversidad, agrupamiento y asociación, en un
bosque tropical lluvioso.\\  }



\author{\Large Marcos Antonio González Reyes\vspace{0.05in} \newline\normalsize\emph{Estudiante, Universidad Autónoma de Santo Domingo (UASD)}  }


\date{}

\usepackage{titlesec}

\titleformat*{\section}{\normalsize\bfseries}
\titleformat*{\subsection}{\normalsize\itshape}
\titleformat*{\subsubsection}{\normalsize\itshape}
\titleformat*{\paragraph}{\normalsize\itshape}
\titleformat*{\subparagraph}{\normalsize\itshape}

\titlespacing{\section}
{0pt}{36pt}{0pt}
\titlespacing{\subsection}
{0pt}{36pt}{0pt}
\titlespacing{\subsubsection}
{0pt}{36pt}{0pt}





\newtheorem{hypothesis}{Hypothesis}
\usepackage{setspace}

\makeatletter
\@ifpackageloaded{hyperref}{}{%
\ifxetex
  \PassOptionsToPackage{hyphens}{url}\usepackage[setpagesize=false, % page size defined by xetex
              unicode=false, % unicode breaks when used with xetex
              xetex]{hyperref}
\else
  \PassOptionsToPackage{hyphens}{url}\usepackage[unicode=true]{hyperref}
\fi
}

\@ifpackageloaded{color}{
    \PassOptionsToPackage{usenames,dvipsnames}{color}
}{%
    \usepackage[usenames,dvipsnames]{color}
}
\makeatother
\hypersetup{breaklinks=true,
            bookmarks=true,
            pdfauthor={Marcos Antonio González Reyes (Estudiante, Universidad Autónoma de Santo Domingo (UASD))},
             pdfkeywords = {Arecaceae, BCI, Lago Gatún, Panamá},  
            pdftitle={La familia Arecaceae en la parcela permanente de 50ha, Barro Colorado
Island.\\
Ecología de comunidades, biodiversidad, agrupamiento y asociación, en un
bosque tropical lluvioso.\\},
            colorlinks=true,
            citecolor=blue,
            urlcolor=blue,
            linkcolor=magenta,
            pdfborder={0 0 0}}
\urlstyle{same}  % don't use monospace font for urls

% set default figure placement to htbp
\makeatletter
\def\fps@figure{htbp}
\makeatother

\usepackage{pdflscape} \newcommand{\blandscape}{\begin{landscape}}
\newcommand{\elandscape}{\end{landscape}} \usepackage{float}
\floatplacement{figure}{H}
\newcommand{\beginsupplement}{ \setcounter{table}{0} \renewcommand{\thetable}{S\arabic{table}} \setcounter{figure}{0} \renewcommand{\thefigure}{S\arabic{figure}} }


% add tightlist ----------
\providecommand{\tightlist}{%
\setlength{\itemsep}{0pt}\setlength{\parskip}{0pt}}

\begin{document}
	
% \pagenumbering{arabic}% resets `page` counter to 1 
%
% \maketitle

{% \usefont{T1}{pnc}{m}{n}
\setlength{\parindent}{0pt}
\thispagestyle{plain}
{\fontsize{18}{20}\selectfont\raggedright 
\maketitle  % title \par  

}

{
   \vskip 13.5pt\relax \normalsize\fontsize{11}{12} 
\textbf{\authorfont Marcos Antonio González Reyes} \hskip 15pt \emph{\small Estudiante, Universidad Autónoma de Santo Domingo (UASD)}   

}

}








\begin{abstract}

    \hbox{\vrule height .2pt width 39.14pc}

    \vskip 8.5pt % \small 

\noindent Sobre la famila Arecaceae en la parcela de 50ha se conoce muy poco de su
ecología, diversidad, asociación (entre especies y variables ambientales
y de suelo, especie/especie, y sitios/especies), y como se agrupan los
sitios en base a sus características. Con este trabajo se trata de
resolver la problemática planteada anteriormente. En base a los datos
obtenidos se concluye que \emph{Chamaedorea tepejilote} y \emph{Bactris
barronis} son las dos especies que no se asocian con otras especies, se
presentan asociaciones entre variables geomorfologicas y entre variables
de riqueza-abundancia. La abundancia de la familia Arecaceae se
encuentra asociada con la riqueza de especies de la familia, la
orientación media y la riqueza global de especies de la parcela
permanente. La riqueza de la familia se asocia a su vez con la variable
geomorfológica de pico. Mediante técnicas de agrupamiento se lograron
alcanzar 5 grupos de sitios mediante la técnica Ward. Utilizando el
Coeficiente de correlación biserial puntual se observa que el grupo dos
de la técnica de agrupamiento Complete mantiene a \emph{Socratea
exorrizha} como especie indicadora. El I de Moran con la transformada de
Hellinger con y sin tendencia, muestra que \emph{Bactris major} tiene
una dependencia espacial con ciertas variables, como Ca, B y Mg, donde
bajos valores de estas variables llevan a que haya abundancia
significativa de esta especie en sus sitios de muestreo. Observar como
el material estudiado se estructura dentro del área muestreada nos
ayudaría a tener una idea de como estarían conformadas las comunidades
de esta familia dentro de otras áreas de estudio que queramos analizar
en el futuro. Los estudios y análisis empleados en este trabajo son
extrapolables a cualquier sitio, y probablemente el presente trabajo
establezca un precedente para investigadores que deseen realizar este
tipo de investigación.


\vskip 8.5pt \noindent \emph{Keywords}: Arecaceae, BCI, Lago Gatún, Panamá \par

    \hbox{\vrule height .2pt width 39.14pc}



\end{abstract}


\vskip 6.5pt


\noindent  \section{Introducción}\label{introducciuxf3n}

Trabajos realizados en esta localidad han estudiado el cambio en la
composición del bosque como resultado de sequías (Valencia \& Balslev,
1997), . La riqueza de especies de plantas está fuertemente asociada a
temperatura, radiación solar y potencial de evapotranspiración, y
muestra una baja relación con la precipitación (Currie, 1991). Aunque el
número absoluto de especies puede cambiar con el tiempo, debido a
eventos de especiación, extinción o dispersión, la persistencia de
patrones predecibles nos dice que tales eventos y sus consecuencias
están de alguna manera limitados geográficamente, sugieren que si
entendiéramos los factores que gobiernan los patrones geográficos, sería
posible predecir la riqueza actual donde faltan datos empíricos, así
como modelar cómo los patrones podrían cambiar con el tiempo dados los
cambios (pasado / futuro) en los factores de control (O'Brien, 1998).

Barro Colorado es una isla localizada en el lago Gatún del Canal de
Panamá. Es un área protegida, en la cual se estudian los bosques
tropicales y que junto con otras cinco penínsulas cercanas, forma el
Monumento Natural Barro Colorado. Fue estructurado en 1923 y está
administrado por el Instituto Smithsoniano desde 1946.

En este estudio se estará trabajando con la familia de plantas
Arecaceae, comúnmente conocidas como palmeras. Las palmas o palmeras son
plantas monocotiledóneas y presentan un crecimiento apical, que al
perder esta parte, detienen su crecimiento, secan y mueren. Las palmas
cuentan con aproxiamadamente 3,500 especies, en su mayoría representada
por árboles, también se encuentran arbustos y enredaderas. Las formas
arbóreas no presentan ramificación en el tronco y tienen una corona
terminal en hojas, generalmente llamadas frondas, que emergen de una en
una de la yema apical. La lignina de sus paredes celulares le confiere
al tallo gran rigidez, y como monocotiledóneas no poseen crecimiento
secundario. Algunas de ellas tienen valor económico y pueden se fuente
de alimento, refugio, ropa y combustible (Glimn-Lacy \& Kaufman, 2006).
Dentro del área, la familia cuenta con cuatro tribus: Iriarte (1
especie), Cocoseae (6), Areceae (1), y Chamaedeeae (1).

Los bosques tropicales de tierra firme de tierras bajas poseen mayor
riqueza, mientras que los bosques pantanosos o crecen en suelos arenosos
o en áreas con clima estacional tienen muchas menos especies; las
comunidades de palmeras en la Amazonía centro-occidental y en América
Central son significativamente más ricas que la región promedio y las
del Caribe son significativamente más pobres en especies, las palmeras
son a menudo especies clave de los ecosistemas y forman conjuntos
complejos de diferentes formas de crecimiento coexistentes que van desde
pequeños arbustos hasta árboles altos y lianas (Balslev et al., 2011).

En el presente trabajo se realizarán estudios, a fin de conocer la
estructura ecológica de la familia, para ello se establecen algunas
preguntas que ayuden a dirigir y enfocar esta investigación:

-¿Dentro de la familia estudiada se presentan especies, sitios, y
variables asociadas?,

-¿Cómo se organizan o agrupan los sitios en el área muestreada?,

-¿Existe alguna relación entre estos grupos y variables/atributos?,

-¿Hay especies indicadoras o con preferencia por determinadas
condiciones ambientales/atributos?,

-¿Está suficientemente representada mi familia, según la diversidad
presentada?,

-¿Existe asociación de la diversidad alpha con variables
ambientales/atributos?, ¿Con cuáles?,

-¿Existe contribución local o por alguna especie a la diversidad beta?,

-¿Alguna(s) especies de mi familia presenta(n) patrón aglomerado, y de
ser así, cuáles?, ¿Dicho patrón se asocia con alguna variable?,

-¿Predicen bien la ocurrencia de dicha(s) especie(s) los modelos de
distribución de especies (SDM)?

\section{Metodología}\label{metodologuxeda}

\subsection{Área de estudio}\label{uxe1rea-de-estudio}

El área de estudio fue la parcela permanente de 50ha en Isla Barro
Colorado, la parcela (en lo adelante BCI), abarca
0.5~km\textsuperscript{2} de los 54~km\textsuperscript{2} que comprende
la isla, cuenta con al menos 315 especies identificadas (Condit, 1998),
distribuidas en 88 familias. Forma parte de ``\emph{The Center for
Tropical Science}'', una red compuesta aproximadamente 15 países, que
estudian los bosques tropicales y la metología estandarizada en grandes
parcelas permanentes, siendo Barro Colorado la primer gran parcela en
ser establecida, censada por primera vez en los años 1981-1983 (Condit,
1998), localizada en el Lago Gatún en Panamá (Hubbell, Condit, \&
Foster, 2005) (ver figura \ref{fig:mapa_cuadros}).

Este tipo de área de estudio se establece con el fin de recolectar y
analizar datos ecológicos para monitorear la dinámica de poblaciones y
la diversidad en estaciones de trabajo permanentes a largo plazo. Esto
se conoce como ``\emph{Long-Term Monitoring}'', puede ser definido como
el levantamiento de datos durante determinado período de tiempo en áreas
contaminadas o con un alto índice de pérdida de especies. En palabras de
Lindenmayer \& Likens (2010), son las mediciones empíricas repetidas
basadas en el campo, se recopilan continuamente y luego se analizan
durante al menos 10 años. Es un estudio a largo plazo cuando documenta
los procesos importantes que componen el ecosistema o el tiempo de
generación del organismo dominante, así, su duración se mide con la
velocidad dinámica del sistema que se está estudiando (Franklin, 1989).

\begin{figure}
\centering
\includegraphics[width=0.50000\textwidth]{mapa_cuadros.png}
\caption{Área de estudio, parcela de 50ha, Isla Barro Colorado.
\label{fig:mapa_cuadros}}
\end{figure}

\subsection{Materiales y métodos}\label{materiales-y-muxe9todos}

En el presente estudio trabajamos con la familia Arecaceae Schultz Sch.,
dentro de la parcela presenta 9 especies y 7 géneros. El estudio se
fundamentó en el uso del censo \#7 de BCI, tomando en cuenta una matriz
de comunidad que contiene los datos relevantes de las especies censadas.
Según (Condit, 1998), la metodología del censo utiliza como criterio de
exclusión, aquellos individuos con menos de 1~cm de diámetro a la altura
del pecho (DAP). Los censos, por lo general, se actualizan cada 5 años.
Para las labores de monitoreo de la inmigración, extinción y reemplazo
en la parcela, los individuos se identifican con un código único, de
forma que los datos sirvan para evluar, longitudinalmente, si hay nuevas
especies, así como para actualizar la abundancia en cada momento censal.

\subsubsection{Medición de
Asociación}\label{mediciuxf3n-de-asociaciuxf3n}

En este análisis nos basamos en los modos Q y R, en tanto que realizamos
comparaciones entre sitios y entre especies. En modo Q nos fundamentamos
en la composición de especies y variables ambientales para evaluar la
disimilaridad entre sitios. Utilizamos la distancia euclídea aplicada a
datos de abundancia de una matriz de comunidad transformados por método
``Hellinger'', tanto como por medio de métricas de disimilaridad de
Jaccard y Sorensen para datos transformados de presencia-ausencia, para
evaluar disimilaridad entre sitios según su composición. Del mismo modo
evaluamos disimilaridad entre sitios a través de variables mixtas de la
matriz ambiental en modo Q, utilizando la métrica de Gower.

En modo R se calculó el grado de dependencia presente entre especies
empleando, en la matriz de comunidad transpuesta, tanto la
transformación ``Chi'', a los datos de abundancia para la posterior
estimación de la distancia, que nos permite comparar especies, como la
métrica de Jaccard a datos de presencia-ausencia. Finalmente, evaluamos
el grado de asociación entre las variables ambientales a partir de los
índices de correlación de Pearson y Spearman, aplicados a las variables
cuantativas de la matriz ambiental.

Las representaciones de las matrices de disimilaridad obtenidas, se
realizaron utilizando `mapas de calor', en los que se representaron, con
distintas tonalidades de colores rosa y azul, las distancias obtenidas."

\subsubsection{Análisis de
Agrupamiento}\label{anuxe1lisis-de-agrupamiento}

Se realizaron análisis de agrupamiento utilizando técnicas de
agrupamiento jerárquico, para agrupar según características o
preferencias comúnes entre sitios, aquí se utilizan criterios de enlace
y sus resultados se visualizan en dendrogramas, para cuya elaboración se
usó distancia Euclídea, que es la distancia entre un punto y otro según
variables descriptivas y se usa la distancia mínima. Aquí utlizamos el
agrupamiento ``\emph{Ward}'' de distancia mínima, determinado para 5
grupos de sitios, el objetivo es definir grupos de manera que la suma de
cuadrados se minimice dentro de cada uno de ellos, basándose en el
criterio del modelo lineal de mínimos cuadrados (Borcard, Gillet, \&
Legendre, 2018). También se aplicaron otras pruebas, en este caso ANOVA,
evalúa homogeneidad de medias, no se cumplen muchos de los supuestos
requeridos para esta prueba; y Kruskal-Wallis que evalúa homogeneidad de
medianas; para calcular asociación entre variables ambientales y de
suelo, y los grupos formados (Batlle, 2020).

Para la generación de dendrogramas se utilizó la función `hclust' del
paquete `stats', presente por defecto en R. A su vez se utilizó
correlación cofenética para determinar el número de grupos de sitios más
apropiado; y `anchura de silueta' para determinar cuantos grupos serían
necesarios o más representativos, así como pruebas multivariadas de
Bootstrap en busqueda de los sitios que presentaran mayores
probabilidades de formar grupos, y la generación de mapas de calor para
representar los grupos de sitios y la distancia entre especies.

\subsubsection{Diversidad Ecológica}\label{diversidad-ecoluxf3gica}

Para calcular el grado de diversidad alpha usamos los grupos generados
mediante el análisis de agrupamiento, `grupos\_ward\_k5'. A fin de
calcular esto se utilizan métodos como la Entropía de Shannon \(H_1\),
que calcula el grado de desorden en la muestra; índice de concentración
de Simpson, que calcula la probabilidad de que dos individuos
seleccionados aleatoriamente puedan ser de la misma especie, Equidad de
Pielou \(J=H_1/H_0\). Números de Hill, riqueza de especies \(N_0=q\),
número de especies abundantes \(N_1=e^H\), y el inverso de Simpson
\(N_1=1/\lambda\); y ratios de Hill \(E_1=N_1/N_0\) (su versión para
equidad Shannon), \(E_2=N_2/N_0\) (versión de equidad de Simpson).

También se usaron modelos con enfoques asintóticos:

-Modelo homogéneo (estándar y MLE), este modelo asume que todas las
especies tienen la misma incidencia o probabilidades de detección.

-Chao1 devolverá una estimación de la riqueza de especies basada en un
vector o matriz de datos de abundancia; y Chao1-bc, una forma corregida
por sesgos para el Chao1.

-iChao1, al igual que el estimador Chao1, el estimador iChao1 es un
límite inferior aproximado para cualquier tamaño de muestra; el
estimador iChao1 es un límite inferior mayor porque siempre es mayor o
igual que el estimador Chao1 (Chiu, Wang, Walther, \& Chao, 2014).

-Basados en ``cobertura'' o ``completitud de muestra''. ACE para datos
de abundancia

-Estimadores Jackknife de primer, utiliza la frecuencia de los únicos
para estimar el número de especies no detectadas; y de segundo orden,
utiliza las frecuencias de ejemplares únicos y duplicados para estimar
el número de especies no detectadas; estiman la riqueza de especies.

Para diversidad beta calculamos la contribución de las especies y los
sitios a la misma, valiéndonos de la función
``\emph{determinar\_contrib\_local\_y\_especie}'', usando la
transformación de Hellinger.

\subsubsection{Análisis de Ecología
Espacial}\label{anuxe1lisis-de-ecologuxeda-espacial}

En ecología espacial usamos la transformada de hellinger y la matriz
ambiental para crear un cuadro de vecindad y ver como se
autocorrelacionan los sitios, se genera un correlograma para las
variables que queremos estudiar mediante la función `sp.correlogram' y
para varias variables como la abundancia de especies y variables
ambientales. También se usaron otros métodos como la prueba Mantel, para
aplicarla a datos de comunidad, es necesario quitar la tendencia
espacial. Para ello, primero hay que ajustar la matriz de comunidad
transformada por Hellinger a la matriz de posiciones XY. El modelo
resultante explicará las abundancias de especies transformadas según la
posición. Los residuos de dicho modelo, contendrán la proporción de las
abundancias transformadas no explicada por la posición. Si dicha
proporción presenta autocorrelación espacial (cuadros de 1 Ha cercanos
entre sí que presentan correlación positiva o negativa), entonces es
probable que se esté frente a un caso de dependencia espacial inducida
por una variable interveniente con matrices de distancia para
autocorrelación espacial con y sin tendencia.

Así mismo, se utilizó el I de Moran, con una matriz de abundancia de
especies transformada sin tendencia que se aplica a variables
ambientales para obtener los datos de autocorrelación y distribución de
especies y variables en los sitios de muestreo.

Para este estudio se utilizó el software estadístico R (R Core Team,
2020), los paquetes vegan (Oksanen et al., 2019), tidyverse (Wickham,
2017), sf (Pebesma, 2018), mapview (Appelhans, Detsch, Reudenbach, \&
Woellauer, 2019), leaflet (Cheng, Karambelkar, \& Xie, 2018), ez
(Lawrence, 2016), psych (Revelle, 2019), adespatial (Dray et al., 2020),
stats (R Core Team, 2020) y scripts del repositorio ``Scripts de
análisis de BCI'' (Batlle, 2020).

\section{Resultados}\label{resultados}

\subsection{Riqueza-abundancia y
presecia-ausencia}\label{riqueza-abundancia-y-presecia-ausencia}

La familia Arecaceae estuvo repesentada por 2,637 individuos, dentro de
los que hubo una riqueza de 9 especies y 7 géneros. La especie más
abundante fue \emph{Oenocarpus mapora} con 1,802 individuos, seguida de
\emph{Socratea exorrizha} (500); \emph{Astrocaryum standleyanum} y
\emph{Bactris major} fueron las de abundancia media con 152 y 112
individuos respectivamente; y \emph{Attalea butyracea}, \emph{Elais
oleifera}, \emph{Bactris coloniata}, \emph{Bactris barronis} y
\emph{Chamaedorea tepejilote} 32, 20, 10, 5 y 4 individuos
respectivamente (ver tabla \ref{tab:abun_sp} y figura
\ref{fig:abun_sp_q}). La mayor abundancia de la familia estuvo presente
en los sitios 5, 6 y 27, con 113, 171 y 136 individuos respectivamente;
y los sitios 35, 40 y 41 los de menor abundancia, presentando 9, 15 y 10
individuos respectivamente (ver figura
\ref{fig:mapa_cuadros_abun_mi_familia}). Los cuadrantes que mayor
riqueza mostraron tuvieron un número de 5 a 6 especies, los que menos
mostraron, tuvieron de 2 a 3 especies, ningún cuadrante mostró estar
representado por solo 1 especie, así mismo ninguno presentó la totalidad
de especies (9) presentes en la parcela (ver figura
\ref{fig:mapa_cuadros_riq_mi_familia}).

\begin{longtable}[]{@{}lr@{}}
\caption{\label{tab:abun_sp}Abundancia de individiuos por especie de la
familia Arecaceae.}\tabularnewline
\toprule
Latin & n\tabularnewline
\midrule
\endfirsthead
\toprule
Latin & n\tabularnewline
\midrule
\endhead
Oenocarpus mapora & 1802\tabularnewline
Socratea exorrhiza & 500\tabularnewline
Astrocaryum standleyanum & 152\tabularnewline
Bactris major & 112\tabularnewline
Attalea butyracea & 32\tabularnewline
Elaeis oleifera & 20\tabularnewline
Bactris coloniata & 10\tabularnewline
Bactris barronis & 5\tabularnewline
Chamaedorea tepejilote & 4\tabularnewline
\bottomrule
\end{longtable}

\begin{figure}
\centering
\includegraphics{manuscrito_files/figure-latex/unnamed-chunk-3-1.pdf}
\caption{\label{fig:abun_sp_q}Abundancia de individuos por especie en
cada cuadrante.}
\end{figure}

\subsection{Análisis de asociación}\label{anuxe1lisis-de-asociaciuxf3n}

Los sitios mostraron grados de correlación estadística mediante análisis
de disimilaridad de distancia Euclídea, de igual manera se econtraron
sitios altamente disímiles del conjunto (sitios 18, 21, 31 y 42)(ver
figura \ref{fig:matriz_disimilaridad_hellinger}).

Para la asociación de especies, los datos arrojaron que
\emph{Chamaedorea tepejilote} y \emph{Bactris barronis} son especies que
no presentan asociaciones con otras; \emph{Socratea exorrizha} se asocia
con \emph{Bactris coloniata}, \emph{Oenocarpus mapora},
\emph{Astrocaryum standleyanum}, \emph{Attalea butyracea} y
\emph{Bactris major}; \emph{Oenocarpus mapora} presenta asociaciones con
las demás especies; el mismo caso anterior se repite con las especies
\emph{Astrocaryum standleyanum} y \emph{Attalea butyracea}; \emph{Elaeis
oleifera} presenta asociaciones con \emph{A. standleyanum},
\emph{Bactris major}, \emph{A. butyracea} y \emph{Oenocarpus mapora};
\emph{Bactris major} carece de asociación con las especies
\emph{Chamaedorea tepejilote}, \emph{Bactris barronis} y \emph{Bactris
coloniata}, (ver figuras \ref{fig:matriz_asociacion_especies} y
\ref{fig:matriz_distancia_especies}).

Se observaron asociaciones entre variables geomorfologicas y variables
de riqueza-abundancia, la abundancia de la familia Arecaceae se encontró
asociada a la variable de pendiente media; y la riqueza de especies
estuvo relacionada con geomorfología de pico, (ver figura
\ref{fig:matriz_correlacion_geomorf_abun_riq_spearman}). En cuanto a
variables de suelo y riqueza-abundancia, la abundancia estuvo asociada
significativamente con las variables B, Ca, Cu, K, Mn, Zn, N, N.min, y
pH de manera negativa, mientras que la riqueza no mostró asociación
significativa con ninguna variable de suelo, (ver figura
\ref{fig:matriz_correlacion_suelo_abun_riq_spearman}).

\subsection{Análisis de
agrupamiento}\label{anuxe1lisis-de-agrupamiento-1}

Se detectaron sitios con altas pobabilidades de formar grupos, cuyo
agrupamiento por método ``\emph{ward}'' resultó en cinco grupos, un gran
grupo con 24 sitios; uno mediano con 15; y tres pequeños con 2, 3 y 6
sitios cada uno, (ver figura \ref{fig:Dendrograma_grupos_ward_5}). A
pesar de la autocorrelación que presentan los sitios entre sí, la
composición de especies de la familia afecta la distribución de los
grupos, haciendo que los sitios formen agrupamientos discontinuos, (ver
figura \ref{fig:mapa_ward_k5}). Este patrón de distribución tiene grados
de correlación con variables ambientales como Al, B, Mn, Cu, Zn, N, pH,
K; y variables geomorfológicas de elevación media, hombrera y pendiente
media, (ver figura \ref{fig:Grupos_Ward_k5_variables}). Los sitios
pertenecientes al grupo cuatro tienen marcada asociación con variables
de suelo como (Potasio, Hierro, Aluminio) mientras que con otras
variables de suelo tiene valores de asociación bajos; con variables
ambientales/geomorfológicas como (Riqueza global, Vertiente,
Heterogeneidad ambiental y Pendiente media) presenta altos grados de
dependencia.

Se observaron especies indicadoras en el agrupamiento ``\emph{ward}''
mediante método del valor indicador (IndVal), y con preferencia de
hábitats con el Coeficiente de Correlación Biserial Puntual (CCBP), las
cuales fueron \emph{Elaeis oleifera}, \emph{Bactris major} y
\emph{Attalea butyracea} en el grupo 4; con la suma de grupos 1+2 se
encuentra \emph{Socratea exorrizha}, y para los grupos 4+5
\emph{Astrocaryum standleyanum}. A su vez, se detectaron especies con
preferancias por hábitats: \emph{Socratea exorrizha} con preferencia por
los habitats ``\emph{OldLow}'' (bosque viejo con relieve bajo) y
``\emph{OldSlope}'' (Bosque viejo con pendiente) en el grupo 2; y
\emph{Elaeis oleifera}, \emph{Bactris major} y \emph{Attalea butyracea}
en el grupo 4, con preferancia por los habitats ``\emph{OldLow}'' y
``\emph{swamp}'' (pantano).

\subsection{Análisis de diversidad}\label{anuxe1lisis-de-diversidad}

Se determinó que la riqueza de la familia Arecaceae (9 especies) en BCI,
está representada en su totalidad, la cobertura del conjunto de datos
fue del 100\% y todos los valores estimados de riqueza fueron del 95\%.
En cuanto a los grupos encontrados: todos los grupos mostraron
completitud de muestra del 100\% y valores estimados de riqueza del
95\%.

En diversidad alpha la Equidad de Pielou y la Ratios de Hill mostraron
asociación estadística con la elevación media y abundancia global. Para
diversidad beta, se encontró que tanto los sitios (13 y 23), como las
especies (\emph{Attalea butyracea}, \emph{Socratea exorrizha} y
\emph{Bactris major}), contribuyen a esta diversidad.

\subsection{Análisis de Ecología
espacial}\label{anuxe1lisis-de-ecologuxeda-espacial-1}

Se detectaron autocorrelaciones mediante el I de Moran, tanto para las
especies, como para las variables geomorfológicas y de suelo. Las
especies que mostraron autocorrelación fueron \emph{Astrocaryum
standleyanum} con valor positivo para los órdenes 1, 2, y negativo para
órdenes 5 y 6; \emph{Elaeis oleifera} y \emph{Oenocarpus mapora} en el
orden 1, positiva y negativa respectivamente; \emph{Socratea exorrizha}
positiva para orden 1 y negativa para orden 5. Las variables
geomorfológicas que tuvieron autocorrelación la presentaron solo para 1
orden, como es el caso de llanura, espolón/gajo, vertiente y vaguada.
Para las variables de suelo las más destacables son: Al con valor
positivo para orden 1 y negativo para orden 7; B, Ca, Zn y K en los
órdenes 1-3 (positiva) y 5-8 (negativa); Mg órdenes 1-3 (positiva) y 6-8
(negativa); Fe positiva en los órdenes (1, 2) y negativa en (7, 8); Cu
en órdenes 1 y 2; Mn positiva para orden 1 y negativa para el 4; P
positiva en órdenes (1, 4) y negativa en los (2, 6) N 1-2 (positiva) y
4-6 (negativa); N.min 1-3 (positiva) y 7-9 (negativa); pH órdenes 1-3
(positiva) y 4-7 (negativa); pendiente y orientación media para orden 1;
y elevación media, positiva para órdenes (1, 2) y negativa para (3, 4).
La distribución de pH en la parcela se presenta de izquierda a derecha,
hacia la izquierda se encuentran los valores de pH más bajos y hacia la
derecha los más altos (ver figura \ref{fig:mapa_cuadros_pH}).

Hubo autocorrelación para el orden 3 en la prueba de Mantel, con
significancia para un valor de 0.001. El I de Moran con la transformada
de Hellinger con y sin tendencia, muestra que \emph{Bactris major} tiene
una dependencia espacial con ciertas variables, como Ca, B y Mg, donde
bajos valores de estas variables llevan a que haya abundancia
significativa de esta especie en sus sitios de muestreo; esta especie se
describe como calciofila (Grandtner \& Chevrette, 2013). \emph{Elaeis
oleifera} parece mostrar preferencia por variables como llanura;
hábitats ``\emph{OldLow}'' y ``\emph{Swamp}''; en presencia de Al esta
especie puede aumentar sus valores de abundancia; y de bajos valores de
N, K, Mg y elvación media. ``\emph{A. butyracea}'' pareciera tener
preferencia por algunas de las variables anteriores. Los sitios del
grupo 4 (13, 18 y 23) son de los que presentan mayor heterogeneidad
ambiental, valores de elevación media similares, cercanos valores Cu, Mn
y Ca. Debido al azar con el que se presentan las especies de la familia
en el área de estudio, estimo que estos modelos no predicen con gran
certeza la distribución u ocurrencia de las mismas.

\begin{figure}
\centering
\includegraphics{matriz_disimilaridad_hellinger.png}
\caption{Matriz de correlacion entre sitios. En color fucsia los sitios
que poseen alta asociación, y en color cian los de baja asociación.
\label{fig:matriz_disimilaridad_hellinger}}
\end{figure}

\begin{figure}
\centering
\includegraphics{matriz_asociacion_especies.png}
\caption{Matriz de asociacion entre especies. Los colores cian
significan nula asociación, y los colores fucsia la asociación que
presentan las especies. \label{fig:matriz_asociacion_especies}}
\end{figure}

\begin{figure}
\centering
\includegraphics{matriz_distancia_especies.png}
\caption{Matriz de distancia entre especies. Los colores cian denotan
larga distancia Euclídea, y los colores fucsia corta distancia.
\label{fig:matriz_distancia_especies}}
\end{figure}

\begin{figure}
\centering
\includegraphics{matriz_correlacion_geomorf_abun_riq_spearman.png}
\caption{Matriz de correlacion de variables geomorfologicas y
abundancia-riqueza.
\label{fig:matriz_correlacion_geomorf_abun_riq_spearman}}
\end{figure}

\begin{figure}
\centering
\includegraphics{matriz_correlacion_suelo_abun_riq_spearman.png}
\caption{Matriz de correlacion de variables suelo y abundancia-riqueza.
\label{fig:matriz_correlacion_suelo_abun_riq_spearman}}
\end{figure}

\begin{figure}
\centering
\includegraphics{Dendrograma_grupos_ward_5.png}
\caption{Dendrograma de grupos. \label{fig:Dendrograma_grupos_ward_5}}
\end{figure}

\begin{figure}
\centering
\includegraphics[width=0.50000\textwidth]{mapa_ward_k5.png}
\caption{Mapa distribución de sitios agrupados.
\label{fig:mapa_ward_k5}}
\end{figure}

\begin{figure}
\centering
\includegraphics{Grupos_Ward_k5_variables.png}
\caption{Gráfico de correlación entre valiables y grupos de sitios.
\label{fig:Grupos_Ward_k5_variables}}
\end{figure}

\section{Discusión}\label{discusiuxf3n}

\emph{Bactris barronis} y \emph{Chamaedorea tepejilote} fueron las
especies que no se relacionaron con otras especies. Cuatro pares de
especies presentan asociación significativa, \emph{Bactris major} y
\emph{Elaeis oleifera}, \emph{Attalea butyracea} y \emph{E. oleifera},
\emph{Oenocarpus mapora} y \emph{Socratea exorrizha}, y
\emph{Astrocaryum standleyanum} con \emph{B. major}. Los sitios
pertenecientes al grupo cuatro tienen marcada asociaión con variables de
suelo como (Potasio, Hierro, Aluminio y pH) mientras que con otras
variables de suelo tiene valores de asociación bajos; con variables
ambientales/geomorfológicas como (Riqueza global, Vertiente,
Heterogeneidad ambiental y Pendiente media) presenta altos grados de
dependencia.

También se determinó que la riqueza de la familia está completamente
representada, con un 100\% de completitud de muestra y covertura del
conjunto, y un 95\% de estimación de riqueza. La diversidad alpha
demostró estar asociada estadísticamente con algunas variables
(elevación media y abundancia global). En la diversidad beta se encontró
la contribución de los sitios (13 y 23); y las especies \emph{A.
butyracea}, \emph{S. exorrizha} y \emph{B. major}.

Las especies, \emph{Astrocaryum standleyanum}, \emph{Elaeis oleifera},
\emph{Oenocarpus mapora}, \emph{Socratea exorrizha}, mostraron la
presencia de autocorrelación espacial. Hubo autocorrelación para algunas
variables geomorfológicas: llanura, espolón/gajo, vertiente y vaguada.
Para las variables de suelo las más destacables son B, Ca, Zn y K en los
órdenes (1-3 y 5-8); Mg órdenes (1-3 y 6-8); N (1-2 y 4-6); N.min (1-3 y
7-9); y pH órdenes (1-7).

La falta de asociaciones de \emph{B. major} y \emph{C. tepejilote} puede
estar determinada por baja abundancia de individuos que presentaron, ya
que son las 2 especies con la más baja abundancia, también podría estar
relacionado con su distribución espacial y algunas variables ambientales
como valores bajos de elevación media; y bajos valores de Al, Fe, Mn, y
bajos valores de pH fomentan la abundancia de \emph{Chamaedorea
tepejilote}.

Los bajos valores de asociación que presenta el grupo 4 con otras
variables de suelo y algunas varibles ambientales/geomorfológicas,
muestran una mejor perspectiva de por qué este grupo presenta esta
distribución y especies muy particulares \emph{Attalea butyracea},
\emph{Bactris major} y \emph{Elaeis oleifera}, especies que no se
presentan en sitios tan contiguos como son los sitios 13, 18 y 23. La
asociación de variables también demostró estar correlacionada con la
distribución de las especies dentro de la parcela.

Los resultados de los análisis de autocorrelación espacial describirían
por qué las especies \emph{A. standleyanum}, \emph{E. oleifera},
\emph{O. mapora}, \emph{S. exorrizha} son las más abundantes,
exceptuando por \emph{Elaeis oleifera}, que es una indicadora
significativa del micro hábitat pantanoso (Legendre \& Condit, 2019) y
reporta un aumento de abundancia de 1 individuo, elevando su abundancia
a 17 individuos para esa fecha, actualmente ese número ha incrementado a
20 individuos.

Estudios de poblaciones de especies de la familia Arecaceae también
reportan que Socratea exorrizha es una especie que parece siempre
presentar una abundancia significativa como reporta (Stevenson \&
Rodríguez, 2008). En las palmeras, se ha destacado la importancia de
variables ambientales: clima, suelos, hidrología y topografía (Balslev
et al., 2011).

Los resultados obtenidos en esta investigación servirán a futuro para
implementar este tipo de metodología en casi cualquier área de estudio
que se proponga investigar, ya que los análisis son extrapolables y
adaptables a casi todos los bosques tropicales, y nos ayudarían a
entender la ecología de sitios que actualmente desconocemos la
estructura de su comunidad, es importante que este tipo de trabajos se
sigan ralizando con frecuencia en todos los bosques del mundo a fin de
observar la dinámica poblacional de distintos ambientes, marcar un punto
de partida en estas áreas desconocidas y ver como las suceciones
ecológicas van cambiando con el tiempo, ver como se desarrollan las
especies e inventariar especies que tal vez no habíamos visto en algunos
sitios debido a distintas razones o variables que tal vez no sabíamos
que podían influir en la biodiversidad en el área muestreada.

\section{Agradecimientos}\label{agradecimientos}

\section{Información de soporte}\label{informaciuxf3n-de-soporte}

\beginsupplement

\begin{figure}
\centering
\includegraphics[width=0.50000\textwidth]{mapa_cuadros_abun_global.png}
\caption{Mapa de abundancia global.
\label{fig:mapa_cuadros_abun_global}}
\end{figure}

\begin{figure}
\centering
\includegraphics[width=0.50000\textwidth]{mapa_cuadros_riq_global.png}
\caption{Mapa de riqueza global. \label{fig:mapa_cuadros_riq_global}}
\end{figure}

\begin{figure}
\centering
\includegraphics[width=0.50000\textwidth]{mapa_cuadros_abun_mi_familia.png}
\caption{Mapa de abundancia de individuos por cuadrante de la familia
Arecaceae. \label{fig:mapa_cuadros_abun_mi_familia}}
\end{figure}

\begin{figure}
\centering
\includegraphics[width=0.50000\textwidth]{mapa_cuadros_riq_mi_familia.png}
\caption{Mapa de riqueza de especies de la familia Arecaceae.
\label{fig:mapa_cuadros_riq_mi_familia}}
\end{figure}

\begin{figure}
\centering
\includegraphics[width=0.50000\textwidth]{mapa_cuadros_ph.png}
\caption{Mapa de pH, parcela de 50ha. \label{fig:mapa_cuadros_pH}}
\end{figure}

\section{\texorpdfstring{\emph{Script}
reproducible}{Script reproducible}}\label{script-reproducible}

\ldots

\section*{Referencias}\label{referencias}
\addcontentsline{toc}{section}{Referencias}

\hypertarget{refs}{}
\hypertarget{ref-MapView}{}
Appelhans, T., Detsch, F., Reudenbach, C., \& Woellauer, S. (2019).
\emph{Mapview: Interactive viewing of spatial data in r}. Retrieved from
\url{https://CRAN.R-project.org/package=mapview}

\hypertarget{ref-balslev2011species}{}
Balslev, H., Kahn, F., Millan, B., Svenning, J.-C., Kristiansen, T.,
Borchsenius, F., \ldots{} Eiserhardt, W. L. (2011). Species diversity
and growth forms in tropical american palm communities. \emph{The
Botanical Review}, \emph{77}(4), 381--425.

\hypertarget{ref-jose_ramon_martinez_batlle_2020_4402362}{}
Batlle, J. R. M. (2020). biogeografia-master/scripts-de-analisis-BCI:
Long coding sessions (Version v0.0.0.9000).
\url{https://doi.org/10.5281/zenodo.4402362}

\hypertarget{ref-borcard2018numerical}{}
Borcard, D., Gillet, F., \& Legendre, P. (2018). \emph{Numerical ecology
with r}. Springer.

\hypertarget{ref-Leaflet}{}
Cheng, J., Karambelkar, B., \& Xie, Y. (2018). \emph{Leaflet: Create
interactive web maps with the javascript 'leaflet' library}. Retrieved
from \url{https://CRAN.R-project.org/package=leaflet}

\hypertarget{ref-chiu2014improved}{}
Chiu, C.-H., Wang, Y.-T., Walther, B. A., \& Chao, A. (2014). An
improved nonparametric lower bound of species richness via a modified
good--turing frequency formula. \emph{Biometrics}, \emph{70}(3),
671--682.

\hypertarget{ref-condit1998tropical}{}
Condit, R. (1998). \emph{Tropical forest census plots: Methods and
results from barro colorado island, panama and a comparison with other
plots}. Springer Science \& Business Media.

\hypertarget{ref-currie1991energy}{}
Currie, D. J. (1991). Energy and large-scale patterns of animal-and
plant-species richness. \emph{The American Naturalist}, \emph{137}(1),
27--49.

\hypertarget{ref-adespatial}{}
Dray, S., Bauman, D., Blanchet, G., Borcard, D., Clappe, S., Guenard,
G., \ldots{} Wagner, H. H. (2020). \emph{Adespatial: Multivariate
multiscale spatial analysis}. Retrieved from
\url{https://CRAN.R-project.org/package=adespatial}

\hypertarget{ref-franklin1989importance}{}
Franklin, J. F. (1989). Importance and justification of long-term
studies in ecology. In \emph{Long-term studies in ecology} (pp. 3--19).
Springer.

\hypertarget{ref-glimn2006palm}{}
Glimn-Lacy, J., \& Kaufman, P. B. (2006). Palm family (arecaceae).
\emph{Botany Illustrated: Introduction to Plants, Major Groups,
Flowering Plant Families}, 125--125.

\hypertarget{ref-grandtner2013dictionary}{}
Grandtner, M. M., \& Chevrette, J. (2013). \emph{Dictionary of trees,
volume 2: South america: Nomenclature, taxonomy and ecology}. Academic
Press.

\hypertarget{ref-Hubbell2005Barro}{}
Hubbell, S., Condit, R., \& Foster, R. (2005). \emph{Forest census plot
on barro colorado island}.
\url{http://ctfs.si.edu/webatlas/datasets/bci/}.

\hypertarget{ref-EZ}{}
Lawrence, M. A. (2016). \emph{Ez: Easy analysis and visualization of
factorial experiments}. Retrieved from
\url{https://CRAN.R-project.org/package=ez}

\hypertarget{ref-legendre2019spatial}{}
Legendre, P., \& Condit, R. (2019). Spatial and temporal analysis of
beta diversity in the barro colorado island forest dynamics plot,
panama. \emph{Forest Ecosystems}, \emph{6}(1), 1--11.

\hypertarget{ref-lindenmayer2010science}{}
Lindenmayer, D. B., \& Likens, G. E. (2010). The science and application
of ecological monitoring. \emph{Biological Conservation}, \emph{143}(6),
1317--1328.

\hypertarget{ref-VeganPack}{}
Oksanen, J., Blanchet, F. G., Friendly, M., Kindt, R., Legendre, P.,
McGlinn, D., \ldots{} Wagner, H. (2019). \emph{Vegan: Community ecology
package}. Retrieved from \url{https://CRAN.R-project.org/package=vegan}

\hypertarget{ref-o1998water}{}
O'Brien, E. (1998). Water-energy dynamics, climate, and prediction of
woody plant species richness: An interim general model. \emph{Journal of
Biogeography}, \emph{25}(2), 379--398.

\hypertarget{ref-sfpackage}{}
Pebesma, E. (2018). Simple Features for R: Standardized Support for
Spatial Vector Data. \emph{The R Journal}, \emph{10}(1), 439--446.
\url{https://doi.org/10.32614/RJ-2018-009}

\hypertarget{ref-RSoft}{}
R Core Team. (2020). \emph{R: A language and environment for statistical
computing}. Retrieved from \url{https://www.R-project.org/}

\hypertarget{ref-psych}{}
Revelle, W. (2019). \emph{Psych: Procedures for psychological,
psychometric, and personality research}. Retrieved from
\url{https://CRAN.R-project.org/package=psych}

\hypertarget{ref-stevenson2008determinantes}{}
Stevenson, P. R., \& Rodríguez, M. E. (2008). Determinantes de la
composición florística y efecto de borde en un fragmento de bosque en el
guaviare, amazonía colombiana. \emph{Colombia Forestal}, \emph{11},
5--17.

\hypertarget{ref-valenciaestudios}{}
Valencia, R., \& Balslev, H. (1997). \emph{ESTUDIOS sobre diversidad y
ecologia de plantas}.

\hypertarget{ref-Tidyverse}{}
Wickham, H. (2017). \emph{Tidyverse: Easily install and load the
'tidyverse'}. Retrieved from
\url{https://CRAN.R-project.org/package=tidyverse}




\newpage
\singlespacing 
\end{document}
